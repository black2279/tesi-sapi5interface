\chapter{Introduzione} %------------------------------ INTRODUZIONE
Lo stage è considerato come l'atto conclusivo della carriera universitaria di uno studente del Corso di Informatica dell'Università di Padova.\\
L'obiettivo dello stage è quello di introdurre lo studente nel mondo del lavoro e di dargli la possibilità di mettere in atto la sua esperienza e le conoscenze acquisite durante gli anni.\\
Il progetto di stage che adesso andremo ad introdurre è stato reso possibile grazie all'incontro avvenuto tra me e il CEO di Mivoq S.r.l Giulio Paci durante l'evento StageIT tenutosi nell'Aprile 2016.\\
Dopo vari incontri e telefonate nel settembre 2016 è stato delineato il progetto di stage.
Nell'ottobre 2016 è stato elaborato in congiunta con il tutor aziendale Giulio Paci il piano di lavoro che successivamente è stato approvato dal relatore Paolo Baldan.\\
Dopo aver concluso l'iter burocratico è stato possibile iniziare lo stage.\\    
In questo capitolo si andranno a descrivere lo scopo del progetto di stage, il prodotto finale, le motivazioni personali che mi hanno spinto ad affrontare questo stage e l'azienda ospitante.\\
Questa introduzione vuole quindi descrivere lo stato iniziale e l'obiettivo a cui si vuole arrivare.
\thispagestyle{empty}
\newpage
\section{Scopo dello stage}
Lo stage si occupa principalmente della tecnologia \gls{tts}. Essa è molto diffusa al giorno d'oggi e la troviamo in ogni sorta di dispositivo. Basti pensare che è impiegata nei traduttori simultanei, nella robotica e negli assistenti virtuali.\\ 
In Italia, però vi sono poche realtà che sono impegnate in questo campo e una di queste è Mivoq S.r.l presente a Padova.\\
Mivoq si occupa principalmente di \gls{sintesivocaleg} e concentra il suo lavoro su due rami:  uno è lo sviluppo di \gls{enginettsg} e l'altro è la personalizzazione della \gls{voceg}.\\
Negli ultimi anni, il lavoro di Mivoq si è focalizzato sullo sviluppo di due sistemi di sintesi vocale: uno è MaryTTS e l'altro è Speect.\\
Arrivati a un buon livello di maturazione dei due prodotti, l'azienda ha deciso che era il momento di integrare i due engine TTS con il sistema operativo Microsoft Windows.\\
Fino a quel momento i sistemi di sintesi vocali erano stati testati e sviluppati solo su sistemi operativi \gls{gnulinuxg}.\\
L'obiettivo dello stage era quindi quello di far interagire i sistemi di sintesi vocale MaryTTS e Speect con il sistema operativo Microsoft Windows tramite l'utilizzo della specifica SAPI~5.\\
Lo standard \gls{sapi5}, sviluppato da Microsoft, è un insieme di metodi ed interfacce che, una volta implementati, permettono di far interagire il sistema operativo con gli engine TTS.\\
Il risultato finale che si vuole ottenere è la creazione di due moduli che rispettino lo standard e forniscano buona parte delle funzionalità messe a disposizione dagli engine TTS al sistema operativo.

\section{Azienda Ospitante}
Mivoq S.r.l è nata nel 2013, ha sede a Padova e si occupa di tecnologie vocali.\\
Il campo in cui è coinvolta maggiormente è il Text-To-Speech. Uno dei prodotti di maggior successo dell'azienda è la personalizzazione della voce, che consiste nel trasformare la voce reale di una persona in una voce digitale. Le voci digitali create hanno come fine quello di essere utilizzate nei prodotti aziendali.
Il processo di personalizzazione della voce è reso possibile grazie alle conoscenze acquisite dai componenti presenti all'interno dell'azienda che lavorano nell'ambito delle tecnologie vocali da anni.
\begin{figure}[H]
\centering
\includegraphics{images/logo-mivoq.png}
\caption{Logo di Mivoq S.r.l}
\end{figure}

\section{Motivazione}
Il motivo che mi ha spinto ad accettare questa proposta di stage deriva dall'interesse che nutro nei confronti della sintesi vocale.
L'aspetto che mi colpisce maggiormente di questa tecnologia è l'elaborazione che viene attuata per tradurre il testo in segnale audio.
Anche se lo stage non tratterà in dettaglio questo aspetto, l'altro motivo che mi ha spinto ad affrontarlo è quello di poter svolgere il lavoro di integrazione degli engine TTS per primo, affrontando così tutte le problematiche del caso. 

\section{Struttura del documento}
La struttura del documento si suddividerà nei seguenti capitoli:
\begin{description}
	\item [Introduzione] descrizione dello scopo dello stage, presentazione dell'azienda, motivazioni e descrizione della struttura del documento.
	\item [Pianificazione] presentazione del piano di lavoro, descrizione delle attività che verranno svolte, elenco degli obiettivi da raggiungere e vincoli tecnologici da rispettare.
	\item [Tecnologie utilizzate] descrizione delle tecnologie utilizzate durante lo stage valutando i vantaggi e gli svantaggi.
	\item [Progettazione] descrizione dell'architettura che si vuole utilizzare e delle relazioni tra i componenti.
	\item [Descrizione dei componenti] descrizione dettagliata dei componenti utilizzati che costituiscono il progetto software.
	\item [Verifica e Validazione] resoconto dei test e del collaudo effettuati.
	\item [Conclusioni] valutazioni finali sullo stage e descrizione del risultato ottenuto.
\end{description}
 