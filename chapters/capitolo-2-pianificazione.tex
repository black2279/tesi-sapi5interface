\chapter{Pianificazione} %------------------------------ CHAPTER TITLE
\thispagestyle{empty}

\newpage
\section{Piano di Lavoro}
Prima di iniziare l'attività di stage è stato stilato assieme al tutor aziendale Giulio Paci e al relatore Paolo Baldan il piano di lavoro.\\
Esso si componeva principalmente di due macro-attività:
\begin{itemize}
	\item Studio delle tecnologie TTS
	\item Implementazione delle interfacce SAPI~5 per gli engine MaryTTS e Speect
\end{itemize}

\subsection{Descrizione delle attività}

\begin{description}[style=unboxed]
	\item[Studio delle tecnologie TTS] Questa attività serviva ad affrontare lo studio delle tecnologie TTS proposte. Gli ambiti di studio coinvolti erano:
	\begin{itemize}
		\item SAPI~5
		\item MaryTTS
		\item Speect
	\end{itemize}
	Per quanto riguarda lo studio di SAPI~5 si trattava di consultare la documentazione messa a disposizione dalla Microsoft Developer Network (MSDN) e di trovare le best practice per l'implementazione dell'interfaccia.\\
	Per MaryTTS e Speect il compito non si fermava solamente allo studio del loro funzionamento, ma bisognava anche renderli funzionanti nel sistema ospitante ovvero Microsoft Windows.\\
	I compiti da svolgere erano: rendere operativi i due engine per gli utenti e fornire ai programmatori futuri delle condizioni di sviluppo idonee.
	
	\item[Implementazione delle interfacce SAPI~5 per gli engine MaryTTS e Speect]
	Questa attività serviva ad implementare le interfacce SAPI~5. L'attività di sviluppo comprendeva oltre alla stesura del codice anche la progettazione, il testing e il debug.
	L'implementazione delle due interfacce risultava diversa nella parte di comunicazione tra l'interfaccia e gli engine.\\
	MaryTTS doveva comunicare tramite chiamate HTTP con l'interfaccia SAPI~5, invece Speect tramite chiamate a funzioni C.  
\end{description}

\subsection{Preventivo}

La suddivisione delle ore concordata insieme al tutor aziendale Giulio Paci e al relatore Paolo Baldan è la seguente:

\begin{center}
	
	\begin{tabular} {|l|l|c p{10cm}|}
		\hline
		\multicolumn{2}{|l|}{\textbf{Durata in ore}}		&	\multicolumn{2}{l|}{\textbf{Descrizione dell'attività}}\\
		\hline
		\multicolumn{2}{|l|}{80}	&	\multicolumn{2}{l|}{Studio delle tecnologie TTS}\\
		\hline
		\multirow{3}{1cm}{ }    &            40            &            \hspace{5mm}•\hspace{2mm}            &  Specifiche TTS SAPI~5\\
		\cline{2-2}
		&            40            &            \hspace{5mm}•\hspace{2mm}            &            Setup di Speect e MaryTTS\\
		\hline
		
		\multicolumn{2}{|l|}{240}	&	\multicolumn{2}{l|}{Implementazione Engine TTS SAPI~5}\\
		\hline
		
		\multirow{3}{1cm}{ }    &            40            &            \hspace{5mm}•\hspace{2mm}            & Creazione di un esempio minimale di engine TTS SAPI~5 utilizzando CURL \\
		\cline{2-2}
		&            80            &            \hspace{5mm}•\hspace{2mm}            &            Creazione di un esempio minimale di engine TTS SAPI~5 utilizzando Speect \\
		\cline{2-2}
		&            60            &            \hspace{5mm}•\hspace{2mm}            &            Implementazione di test (integrazione e unità) per engine TTS SAPI~5 \\
		\cline{2-2}
		&            60            &            \hspace{5mm}•\hspace{2mm}            &            Debug del codice: 2 settimane \\
		\hline
	\end{tabular}
	
\end{center}

\subsection{Prodotto finale}
Lo stage ha come obiettivo finale la creazione di un prodotto.
Il risultato che si vuole ottenere è l'integrazione degli engine TTS MaryTTS e Speect con il sistema operativo Microsoft Windows. Tutto ciò dovrà essere ottenuto grazie all'implementazione delle interfacce SAPI~5.
L'utilizzo della specifica SAPI~5 renderà possibile la fruizione degli engine TTS a livello di sistema operativo, questo significa che le applicazioni utente potranno essere sia Windows stesso che altre applicazioni come ad esempio lettori, avatar o assistenti vocali.\\
Le interfacce dovranno fornire la possibilità di controllare dei parametri della voce che dovranno essere:
\begin{itemize}
	\item la velocità;
	\item la tonalità (o pitch);
\end{itemize}


\subsection{Obiettivi}

Per definire in modo ottimale le priorità sul prodotto si sono estratti dei requisiti minimi per soddisfare l'esigenza del committente.\\
Questi requisiti sono stati in divisi in categorie e sono identificati dalle seguenti sigle:
\begin{itemize}
	\item \textbf{ob} per i requisiti obbligatori, vincolanti in quanto obiettivo primario
	richiesto dal committente;
	\item  \textbf{de} per i requisiti desiderabili, non vincolanti o strettamente necessari,
	ma dal riconoscibile valore aggiunto;
	\item \textbf{op} per i requisiti opzionali, rappresentanti valore aggiunto non
	strettamente competitivo.
\end{itemize}
Le sigle precedentemente indicate saranno seguite da una coppia sequenziale di numeri, identificativo del requisito.\\

Si prevede lo svolgimento dei seguenti obiettivi:
\begin{itemize}
	\item \textbf{Obbligatori}
	\begin{itemize}
		\item \textbf{ob01}: Implementazione di un engine TTS SAPI~5 funzionante mediante chiamate HTTP a MaryTTS;
		\item \textbf{ob02}: Implementazione di un engine TTS SAPI~5 funzionante mediante chiamate C a Speect;
		\item \textbf{ob03}: Implementazione di test di integrazione per gli engine TTS SAPI~5;
	\end{itemize}
	
	\item \textbf{Desiderabili} 
	\begin{itemize}
		\item \textbf{de01}: Implementazione di un engine TTS SAPI~5 con supporto per cambio di velocità e cambio pitch;
		\item \textbf{de02}: Implementazione di test di unità per Speect;
	\end{itemize}
	
	\item \textbf{Opzionali}
	\begin{itemize}
		\item \textbf{op01}: Implementazione di un engine TTS SAPI~5 con pieno supporto a tutte le funzionalità;
		\item \textbf{op02}: Configurazione di una voce inglese per Speect.
	\end{itemize}
\end{itemize}

\subsection{Vincoli Tecnologici}
Per soddisfare i requisiti del prodotto sono stati imposti alcuni vincoli tecnologici.\\
Le tecnologie che dovranno essere utilizzate durante lo stage sono:
\begin{itemize}
	\item Linguaggio di programmazione \textbf{C} o \textbf{C++};
	\item Ambiente di sviluppo \textbf{Visual Studio};
	\item Sistema operativo \textbf{Microsoft Windows};
	\item Libreria \textbf{CURL} per eseguire per le chiamate HTTP;
	\item Specifica \textbf{SAPI~5};
	\item Engine TTS \textbf{MaryTTS} e \textbf{Speect}.
\end{itemize}

