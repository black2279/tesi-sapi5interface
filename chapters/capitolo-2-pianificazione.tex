\chapter{Pianificazione} %------------------------------ CHAPTER TITLE
La durata dello stage è stimata attorno alle 320 ore circa, senza una buona pianificazione è quasi impossibile portare a termine un progetto.\\
Per questo motivo è stato definito un piano di lavoro che verrà successivamente descritto.\\
Lo stage è stato diviso in attività. 
Ad ognuna è stata associata una stima temporale entro la quale si presume possa concludersi. Per definire le tempistiche servono tanta esperienza e la consapevolezza delle risorse a disposizione.\\
Un'attenzione particolare deve essere posta nei confronti dei rischi che si possono incontrare e prevenire le loro possibili conseguenze a priori.\\
Assieme alle attività da svolgere vengono definiti inoltre gli obiettivi che si vogliono rispettare in modo da avere dei feedback sul lavoro svolto.\\
Il progetto di stage non ha potuto sfruttare qualsiasi tecnologia a disposizione perchè sono stati imposti dei vincoli da rispettare che verranno illustrati in questo capitolo. 

\thispagestyle{empty}

\newpage
\section{Piano di Lavoro}
Per ottenere una buona organizzazione delle attività di stage, è stato stilato assieme al tutor aziendale Giulio Paci e al relatore Paolo Baldan il piano di lavoro.\\
Le ore di lavoro erano suddivise principalmente in due macro-attività:
\begin{itemize}
	\item Studio delle tecnologie TTS
	\item Implementazione delle interfacce SAPI~5 per gli engine MaryTTS e Speect
\end{itemize}

\subsection{Descrizione delle attività}

\begin{description}[style=unboxed]
	\item[Studio delle tecnologie TTS] Questa attività serviva ad affrontare lo studio delle tecnologie TTS proposte. Gli ambiti di studio coinvolti erano:
	\begin{itemize}
		\item specifica SAPI~5
		\item MaryTTS
		\item Speect
	\end{itemize}
	Lo studio della specifica SAPI~5 consisteva nel consultare la documentazione messa a disposizione dalla \gls{msdn} e trovare le best practice per l'implementazione dell'interfaccia.\\
	Per quanto riguardava MaryTTS e Speect, il lavoro si divideva in due parti: una era lo studio del loro funzionamento e l'altra consisteva nel renderli operativi all'interno di Microsoft Windows.\\
	
	\item[Implementazione delle interfacce SAPI~5 per gli engine MaryTTS e Speect]
	Questa attività serviva ad implementare le interfacce SAPI~5 e a creare l'infrastruttura necessaria per permettere la comunicazione tra gli engine TTS e la specifica. Le fasi di questa attività erano la progettazione, la codifica, il testing e il debug.\\
	Per ogni engine TTS, però bisognava sviluppare un'implementazione differente. Per MaryTTS il vincolo era l'utilizzo della libreria cURL per effettuare le chiamate \gls{httpg} al server, invece per Speect il vincolo era l'utilizzo delle \gls{api} scritte in C presenti all'interno della libreria dell'engine.
	  
\end{description}

\subsection{Preventivo}

La suddivisione delle ore concordata insieme al tutor aziendale Giulio Paci e al relatore Paolo Baldan era la seguente:

\begin{center}
	
	\begin{tabular} {|l|l|c p{10cm}|}
		\hline
		\multicolumn{2}{|l|}{\textbf{Durata in ore}}		&	\multicolumn{2}{l|}{\textbf{Descrizione dell'attività}}\\
		\hline
		\multicolumn{2}{|l|}{80}	&	\multicolumn{2}{l|}{Studio delle tecnologie TTS}\\
		\hline
		\multirow{3}{1cm}{ }    &            40            &            \hspace{5mm}•\hspace{2mm}            &  Specifiche TTS SAPI~5\\
		\cline{2-2}
		&            40            &            \hspace{5mm}•\hspace{2mm}            &            Setup di Speect e MaryTTS\\
		\hline
		
		\multicolumn{2}{|l|}{240}	&	\multicolumn{2}{l|}{Implementazione Engine TTS SAPI~5}\\
		\hline
		
		\multirow{3}{1cm}{ }    &            40            &            \hspace{5mm}•\hspace{2mm}            & Creazione di un esempio minimale di engine TTS SAPI~5 utilizzando MaryTTS tramite la libreria cURL \\
		\cline{2-2}
		&            80            &            \hspace{5mm}•\hspace{2mm}            &            Creazione di un esempio minimale di engine TTS SAPI~5 utilizzando Speect \\
		\cline{2-2}
		&            60            &            \hspace{5mm}•\hspace{2mm}            &            Implementazione di test (integrazione e unità) per engine TTS SAPI~5 \\
		\cline{2-2}
		&            60            &            \hspace{5mm}•\hspace{2mm}            &            Debug del codice: 2 settimane \\
		\hline
	\end{tabular}
	
\end{center}

\subsection{Problemi incontrati}
Durante le attività di stage non sono sorti gravi problemi che hanno impedito di proseguire il lavoro in maniera lineare.\\
L'ostacolo più grande da superare si è presentato nella seconda parte dello stage, durante lo sviluppo dell'implementazione dell'engine TTS SAPI~5 mediante Speect.\\
Premettendo che, la versione di Speect sviluppata da Mivoq fino a quel momento era resa funzionante solo all'interno del sistema operativo GNU/Linux, era mio compito fare altrettanto per Microsoft Windows.\\
Durante questa attività mi sono imbattuto nella compilazione di un plugin utile per il processo di \gls{postaggingg} e determinante per il funzionamento di Speect.\\
Il plugin utilizzava la libreria \gls{hunposg} che non era mai stata compilata e testata per il sistema Microsoft Windows.
Dopo svariati tentativi e una settimana di lavoro, grazie anche all'aiuto del tutor aziendale, sono riuscito a rendere funzionante il plugin e con esso tutto il sistema di sintesi vocale Speect.\\
Questo momento è stato cruciale per la buona riuscita dello stage e ha confermato l'impiego della libreria HunPos all'interno di Speect per i due sistemi operativi.

\subsection{Prodotto finale}
Lo stage ha come obiettivo finale la creazione di un prodotto.
Il risultato che si vuole ottenere è l'integrazione degli engine TTS MaryTTS e Speect con il sistema operativo Microsoft Windows. Tutto ciò dovrà essere ottenuto grazie all'implementazione delle interfacce SAPI~5.\\
L'utilizzo della specifica SAPI~5 renderà possibile la fruizione degli engine TTS a livello di sistema operativo, questo significa che le applicazioni utente potranno essere sia Windows stesso che altre applicazioni come ad esempio lettori, avatar o assistenti vocali.\\
Le implementazioni ottenute dovranno fornire la possibilità di controllare i seguenti parametri della voce:
\begin{itemize}
	\item la velocità;
	\item la tonalità (o pitch).
\end{itemize}

Ad esempio, se consideriamo un'applicazione utente che segua la specifica SAPI~5 (Balabolka), questa dovrà essere in grado di individuare le voci messe a disposizione dagli engine TTS MaryTTS e Speect ed eseguire la sintesi vocale con i valori dei parametri scelti.


\subsection{Obiettivi}

Per definire in modo ottimale le priorità sul prodotto si sono estratti dei requisiti minimi per soddisfare l'esigenza del committente.\\
Questi requisiti sono stati in divisi in categorie e sono identificati dalle seguenti sigle:
\begin{itemize}
	\item \textbf{ob} per i requisiti obbligatori, vincolanti in quanto obiettivo primario
	richiesto dal committente;
	\item  \textbf{de} per i requisiti desiderabili, non vincolanti o strettamente necessari,
	ma dal riconoscibile valore aggiunto;
	\item \textbf{op} per i requisiti opzionali, rappresentanti valore aggiunto non
	strettamente competitivo.
\end{itemize}
Le sigle precedentemente indicate saranno seguite da una coppia sequenziale di numeri, identificativo del requisito.\\

Si prevede lo svolgimento dei seguenti obiettivi:
\begin{itemize}
	\item \textbf{Obbligatori}
	\begin{itemize}
		\item \textbf{ob01}: Implementazione di un engine TTS SAPI~5 funzionante mediante chiamate HTTP a MaryTTS;
		\item \textbf{ob02}: Implementazione di un engine TTS SAPI~5 funzionante mediante chiamate C a Speect;
		\item \textbf{ob03}: Implementazione di test di integrazione per gli engine TTS SAPI~5;
	\end{itemize}
	
	\item \textbf{Desiderabili} 
	\begin{itemize}
		\item \textbf{de01}: Implementazione di un engine TTS SAPI~5 con supporto per cambio di velocità e cambio pitch;
		\item \textbf{de02}: Implementazione di test di unità per Speect;
	\end{itemize}
	
	\item \textbf{Opzionali}
	\begin{itemize}
		\item \textbf{op01}: Implementazione di un engine TTS SAPI~5 con pieno supporto a tutte le funzionalità;
		\item \textbf{op02}: Configurazione di una voce inglese per Speect.
	\end{itemize}
\end{itemize}

\subsection{Vincoli Tecnologici}
Per soddisfare i requisiti del prodotto sono stati imposti alcuni vincoli tecnologici.\\
Le tecnologie che dovranno essere utilizzate durante lo stage sono:
\begin{itemize}
	\item Linguaggio di programmazione \textbf{C} o \textbf{C++};
	\item Ambiente di sviluppo \textbf{Visual Studio};
	\item Sistema operativo \textbf{Microsoft Windows};
	\item Libreria \textbf{CURL} per eseguire per le chiamate HTTP;
	\item Specifica \textbf{SAPI~5};
	\item Engine TTS \textbf{MaryTTS} e \textbf{Speect}.
\end{itemize}

