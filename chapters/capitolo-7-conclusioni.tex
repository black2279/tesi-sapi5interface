\chapter{Conclusioni}
\thispagestyle{empty}

\newpage

\section{Risultato ottenuto}
Gli obiettivi posti prima dello stage sono stati raggiunti, ottenendo così il funzionamento degli engine TTS Speect e MaryTTS tramite la specifica SAPI~5 all'interno del sistema operativo Microsoft Windows.\\
Questo stage ha dato conferma che l'engine TTS Speect e tutti i suoi plugin possono essere sviluppati e resi funzionanti all'interno del sistema operativo Microsoft Windows.\\
Inoltre, questo progetto ha messo in pratica ciò che era stato studiato tempo prima da parte del tutor aziendale e ha creato così il punto di partenza per integrare le tecnologie vocali dell'azienda nel sistema operativo Windows.

\section{Sviluppi futuri}
Questo progetto come detto in precedenza ha fornito un punto di partenza per sviluppi futuri.\\
Adesso andremo ad elencare alcuni possibili scenari:
\begin{itemize}
	\item \textbf{Integrazione completa con assistente vocale di Windows} Uno possibile sviluppo di questo progetto potrebbe essere l'integrazione completa degli engine TTS a disposizione con l'assistente vocale di Windows. Questa possibilità viene già fornita tramite questo progetto, ma deve essere ancora perfezionata per fornire il completo supporto.
	Seguendo questa strada, in un prossimo futuro, grazie al contributo dell'azienda, sarà possibile utilizzare, ad esempio, la propria voce all'interno dell'assistente vocale.
	\item \textbf{Possibilità di utilizzare le voci personalizzate} Un altro possibile sviluppo è quello di estendere le funzionalità messe a disposizione da questo progetto fino al punto di sfruttare al massimo le capacità degli engine TTS.
	Ad esempio, sarà possibile usufruire delle voci personalizzate create dall'azienda con un alto livello di customizzazione.
\end{itemize}

\section{Considerazioni finali}
A stage concluso posso dire che mi sento molto soddisfatto del lavoro svolto. Credo sia stato molto istruttivo e mi abbia fatto acquisire nuove conoscenze.
Il tutor aziendale e le persone che lavorano in azienda sono state molto disponibili fin da subito.
Questo stage è stato molto importante per me, perchè mi ha dato modo di conoscere il mondo del lavoro e come ci si comporta in determinate occasioni.
Lo stage è stato anche il luogo in cui ho avuto la possibilità di portare a compimento un progetto in completa autonomia.
Durante lo stage ho cercato sempre di risolvere i problemi incontrati durante lo sviluppo del progetto in maniera autonoma, solamente in alcuni casi particolari richiedevo aiuto al tutor aziendale.\\
Dal punto di vista delle conoscenze acquisite lo stage mi ha dato modo di approfondire alcuni aspetti dei linguaggi C e C++, conoscere l'interfaccia SAPI~5, gli engine TTS MaryTTS e Speect, alcuni aspetti della fonetica e della linguistica e mi ha fatto conoscere l'esistenza di molti strumenti, a me sconosciuti, utili per lo sviluppo delle applicazioni.\\
Lo stage mi avvicinato molto al mondo della sintesi vocale e mi ha fatto apprezzare il lavoro e l'impegno che mette l'azienda nel portare avanti i propri progetti.\\
Spero, in un prossimo futuro, di mettere in atto l'esperienza e le conoscenze acquisite in altri progetti. 
 