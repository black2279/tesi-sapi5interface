\chapter{Tecnologie Utilizzate} %------------------------------ CHAPTER TITLE
Il progetto di stage ha comportato l'utilizzo di molte tecnologie. Buona parte di esse sono state scelte dettate dai vincoli tecnologici imposti, altre sono state utilizzate per ottimizzare e facilitare il lavoro.\\
Tutte le tecnologie impiegate sono state visionate e approvate dal tutor aziendale in modo da non prendere strade troppo ostiche od obsolete.\\
In questo capitolo verranno descritte le tecnologie utilizzate dal progetto di stage, il loro impiego e verranno analizzati i vantaggi e gli svantaggi che esse possono portare. 
\thispagestyle{empty}

\newpage
\section{Balabolka}
Balabolka è un applicazione desktop freeware disponibile per il sistema operativo Microsoft Windows che svolge la funzione di lettore. Essa offre la possibilità di leggere dei testi selezionando una voce a disposizione.\\
Implementa quasi tutti i metodi lato utente dell'interfaccia SAPI~5, supportando in primo luogo la sintesi, la tracciatura delle parole lette e il controllo dei vari parametri della voce.\\
Questa applicazione verrà utilizzata sia in fase di test che in fase di debug.
\subsection*{Vantaggi}
\begin{itemize}
	\item implementa diverse interfacce per tecnologie TTS tra cui SAPI~5;
	\item dispone di un'interfaccia chiara e semplice;
	\item l'applicazione è gratuita.
\end{itemize}
\subsection*{Svantaggi}
\begin{itemize}
	\item è disponibile solo la versione a 32-bit.
\end{itemize}

\section{cURL}
cURL è un progetto software open-source utilizzato per comunicare attraverso vari protocolli di rete e viene distribuito in due modalità: uno strumento a linea di comando e una libreria.\\
Durante lo stage cURL verrà impiegato sottoforma di libreria per permettere la comunicazione con il server su cui risiede MaryTTS.
\subsection*{Vantaggi}
\begin{itemize}
	\item compatibile con varie piattaforme;
	\item è scritta in C;
	\item open-source;
	\item è gratuita.
\end{itemize}
\subsection*{Svantaggi}
\begin{itemize}
	\item offre delle API non sempre intuitive.
\end{itemize}
\section{C}
Il C è un linguaggio di programmazione nato nel 1972. Segue i paradigmi di tipo strutturato e procedurale. È considerato un linguaggio di programmazione ad "alto livello" ma integra anche caratteristiche vicine al linguaggio Assembly e al linguaggio macchina.\\
Questo linguaggio sarà utilizzato per manutenere Speect e codificare le funzioni di supporto alle implementazioni delle interfacce SAPI~5.
\subsection*{Vantaggi}
\begin{itemize}
	\item mirato all'efficienza;
	\item la sua struttura modulare facilita il debug, il testing e la manutenzione;
	\item ha un'elevata portabilità.
\end{itemize}
\subsection*{Svantaggi}
\begin{itemize}
	\item non è orientato agli oggetti;
	\item per gli utenti programmatori neofiti la gestione della memoria può essere complicata e alle volte pericolosa.
\end{itemize}
\section{C++}
Il C++ è un linguaggio di programmazione orientato agli oggetti, con tipizzazione statica. È stato sviluppato da Bjarne Stroustrup nel 1983 come un miglioramento del linguaggio C.
Offre molti paradigmi di programmazione che variano dal procedurale al funzionale.
Il paradigma più utilizzato al giorno d'oggi è quello ad oggetti.\\
Il linguaggio C++ verrà utilizzato per implementare l'interfaccia SAPI sotto forma di libreria dinamica (\gls{dllg}).
\subsection*{Vantaggi}
\begin{itemize}
	\item è compatibile con il linguaggio C;
	\item offre un alto livello di astrazione;
	\item è mirato all'efficienza e alle performance.
\end{itemize}
\subsection*{Svantaggi}
\begin{itemize}
	\item non supporta nativamente i thread;
	\item i puntatori, le variabili globali e le funzioni friend lo rendono un linguaggio poco sicuro.
\end{itemize}
\section{Git}
Git è un software di controllo versione creato da Linus Torvalds nel 2005. Questo strumento è stato ideato per facilitare lo sviluppo di progetti software. I due vantaggi principali di questa tecnologia sono: il lavoro può essere svolto da più utenti contemporaneamente e la storia del progetto software può essere gestita in maniera completa.\\
Questo software verrà utilizzato per versionare lo sviluppo dell'engine Speect e dell'interfaccia SAPI.
\subsection*{Vantaggi}
\begin{itemize}
	\item è più veloce rispetto ad altri software di versionamento (esempio: SVN);
	\item permette di lavorare in locale;
	\item è gratuito e open source.
\end{itemize}
\subsection*{Svantaggi}
\begin{itemize}
	\item non è sempre intuitivo, richiede dello studio;
	\item non presenta un sistema automatico per la numerazione delle versioni;
	\item risoluzione dei conflitti a volte complicata.
\end{itemize}
\section{GitLab}
GitLab è una piattaforma web per la gestione di \gls{repositoryg} Git scritta da Dmitriy Zaporozhets e Valery Sizov, lanciata nel 2011. Oltre all'organizzazione delle varie repository, essa fornisce la gestione di pagine wiki e da la possibilità di utilizzare un sistema di \gls{issuetrackingg}.\\
La piattaforma verrà utilizzata per ospitare lo sviluppo del componente SAPI per Windows.
\subsection*{Vantaggi}
\begin{itemize}
	\item numero illimitato di progetti privati e collaboratori con account gratuito;
	\item possibilità di aggiungere sistemi di Continuous Integration;
	\item è possibile attribuire differenti ruoli ai collaboratori per gestire i loro permessi.
\end{itemize}
\subsection*{Svantaggi}
\begin{itemize}
	\item non è aperta ad altri sistemi di controllo versione.
\end{itemize}
\section{GitHub}
GitHub è una piattaforma web nata principalmente per la gestione di repository Git, sviluppata da GitHub Inc. e lanciata nel 2008. Come altri servizi di hosting di codice sorgente, offre uno strumento di issue tracking e la possibilità di creare pagine wiki. Al suo interno è presente anche un altro componente molto utile chiamato Gist che serve a condividere velocemente delle parti di codice. La particolarità di questo componente è che il codice è sottoposto a versionamento.\\
GitHub verrà utilizzato per portare avanti lo sviluppo di Speect e per entrare in possesso della libreria di pos-tagging \gls{hunposg}.
\subsection*{Vantaggi}
\begin{itemize}
	\item numero illimitato di repository pubbliche con account gratuito;
\end{itemize}
\subsection*{Svantaggi}
\begin{itemize}
	\item non è possibile avere repository private con un account gratuito;
	\item non è possibile replicare il servizio su altre macchine.
\end{itemize}
%Jansson
\section{MaryTTS}
MaryTTS è un motore di sintesi vocale open-source scritto in Java. È stato sviluppato inizialmente dalla collaborazione di DFKI e dall'Institute of Phonetics dell'Università di Saarland. La versione che verrà impiegata per condurre il lavoro è stata sviluppata appositamente da Mivoq e prende il nome di Flexible-Adaptive-Text-To-Speech (FATTS). La versione utilizzata presenta delle differenze rispetto all'originale soprattutto a livello di interfaccia e di output messi a disposizione.
\subsection*{Vantaggi}
\begin{itemize}
	\item è gratuito e open source;
	\item ha un'alta portabilità;
	\item l'engine è pensato come un server HTTP;
	\item la versione proposta da Mivoq mette a disposizione delle Web API che rispondono in formato \gls{jsong};
	\item supporta molte lingue.
\end{itemize}
\subsection*{Svantaggi}
\begin{itemize}
	\item le API originali rispondono in un formato non sempre adeguato;
	\item la modifica e l'implementazione di nuove features non è sempre semplice.
\end{itemize}
\section{Microsoft Windows}
Microsoft Windows è un sistema operativo nato nel 1985 e sviluppato dalla Microsoft Corporation. La sua caratteristica principale consiste nella sua interfaccia grafica composta da finestre. A differenza di altri sistemi operativi (ad esempio Unix-like) la parte grafica non è opzionale. Ad oggi, Windows è uno dei sistemi operativi più diffusi sul mercato coprendo dispositivi come PC, smartphone e tablet.\\
Windows è il sistema operativo dove verranno implementate e sviluppate le interfacce SAPI per i due engine di sintesi vocale.
\subsection*{Vantaggi}
\begin{itemize}
	\item è presente su molti dispositivi;
	\item è disponibile un'ampia gamma di software compatibili.
\end{itemize}
\subsection*{Svantaggi}
\begin{itemize}
	\item è a pagamento;
	\item non è open source;
	\item non sempre facilita il lavoro del programmatore.
\end{itemize}
\section{NSIS}
Nullsoft Scriptable Install System (NSIS) è un sistema di installazione guidato da script creato dalla Nullsoft nel 2000. Il suo scopo è quello di automatizzare e facilitare l'installazione di software di terze parti in ambiente Windows. Permette di creare pacchetti di installazione in diverse lingue e utilizzando diversi tipi di compressione.
\subsection*{Vantaggi}
\begin{itemize}
	\item linguaggio di script semplice;
	\item supporta un'installazione multilingua;
	\item è possibile eseguire la cross-compilazione in ambiente POSIX;
	\item interfaccia grafica accattivante;
	\item sono disponibili diversi metodi di compressione.
\end{itemize}
\subsection*{Svantaggi}
\begin{itemize}
	\item è disponibile solo per la piattaforma x86.
\end{itemize}
\section{SAPI~5}
Microsoft Speech API 5 è un insieme di interfacce e di funzioni sviluppate da Microsoft per integrare nel proprio sistema operativo la sintesi vocale e il riconoscimento vocale. La versione verrà utilizzata è la numero 5 perché rispetto alle versioni precedenti non è dipendente dagli engine TTS che si vogliono utilizzare. Riassumendo, SAPI~5 si pone come un middleware che permette di interfacciare applicazioni utente con engine TTS e viceversa. Questo fa si che scrivendo un componente SAPI~5 che si interfacci con un engine TTS possa essere riutilizzato da diverse applicazioni.
\subsection*{Vantaggi}
\begin{itemize}
	\item permette di rendere facilmente usufruibile un engine TTS al sistema operativo;
	\item un'applicazione che implementa un'interfaccia SAPI~5 può utilizzare tutti i servizi vocali che seguono la medesima specifica;
	\item offre un'interfaccia semplice ma allo stesso tempo avanzata.
\end{itemize}
\subsection*{Svantaggi}
\begin{itemize}
	\item la documentazione alle volte risulta imprecisa e approssimativa.
\end{itemize}
\section{Speect}
Speect è un engine TTS multilingua completo, scritto in C e sviluppato dal gruppo Human Language Technologies del Meraka Institute, CSIR, in Sud-Africa a partire dal 2010. Speect è un engine TTS pensato come un sistema a \gls{pluging} per favorire la modularità. Questo aspetto permette di sviluppare e sostituire plugin in modo indipendente, aumentando di conseguenza anche la testabilità.\\
La versione che si andrà ad utilizzare è stata sviluppata da Mivoq e porta con sè alcuni miglioramenti rispetto all'originale.
\subsection*{Vantaggi}
\begin{itemize}
	\item è multilingua;
	\item è open source;
	\item è scritto in ANSI C90;
	\item è portabile;
	\item dispone di un'alta modularità;
	\item ha ampia possibilità di configurazione;
	\item permette di testare i vari plugin separatamente.
\end{itemize}
\subsection*{Svantaggi}
Non stati rilevati svantaggi enormi nell'utilizzo di questo software.   
\section{TTSApplication}
TTSApplication è un'applicazione fornita da Microsoft pensata per testare le varie funzionalità messe a disposizione dalla specifica SAPI~5. Quest'applicazione ha il vantaggio di essere accompagnata dal codice sorgente e quindi può essere modificata e ricompilata. Il codice sorgente è disponibile sulla piattaforma GitHub al seguente indirizzo \url{https://github.com/Microsoft/Windows-classic-samples/tree/master/Samples/Win7Samples/winui/speech/ttsapplication}.
La versione fornita da Microsoft permette di: sintetizzare il testo con qualsiasi voce presente nel sistema operativo, visualizzare gli eventi SAPI che vengono lanciati e animare un semplice avatar attraverso i visemi generati dalla voce.\\
Questo software verrà utilizzato per condurre i test e la fase di debug del componente SAPI che verrà sviluppato. 
\subsection*{Vantaggi}
\begin{itemize}
	\item è open source;
	\item permette di testare la maggior parte delle funzionalità della specifica SAPI 5 riguardante la sintesi vocale;
	\item è scritta in C++;
	\item ha un'interfaccia semplice.
\end{itemize}
\subsection*{Svantaggi}
Non stati rilevati svantaggi enormi nell'utilizzo di questo software.
\section{Visual Studio 2015}
Visual Studio 2015 è un \gls{ide} sviluppato da Microsoft e lanciato nel 2015. Si pone come strumento principale per sviluppare applicazioni per il sistema operativo Windows.\\ 
I linguaggi di programmazione supportati sono:
\begin{itemize}
	\item C;
	\item C++;
	\item C\#;
	\item F\#;
	\item Visual C;
	\item Visual C++;
	\item Visual C\#;
	\item Visual Basic;
	\item Javascript;
	\item Typescript;
\end{itemize}
Visual Studio nel suo pacchetto comprende compilatori, editor, designer e altri strumenti utili in fase di test e di debug. Inoltre possiede sistemi di versionamento integrati tra i quali Git.\\
Offre un'ottima gestione delle configurazioni e una buona organizzazione dei file all'interno dei progetti creati.
\subsection*{Vantaggi}
\begin{itemize}
	\item alcune versioni sono presenti in forma gratuita;
	\item supporta molti linguaggi;
	\item è possibile estendere le sue funzionalità tramite plugin;
	\item sono presenti template di progetto già pronti;
	\item copre lo sviluppo delle piattaforme mobili (Android, iOS, Windows Phone);
	\item possiede un debugger avanzato;
	\item è predisposto per l'esecuzione dei test e la loro gestione.
\end{itemize}
\subsection*{Svantaggi}
\begin{itemize}
	\item è molto oneroso in termini di memoria sul disco, anche optando per una configurazione minima;
	\item il sistema IntelliSense non è sempre preciso;
	\item se si utilizza il compilatore \gls{msvc}, il codice generato ha un'esecuzione più lenta rispetto a quella generata dal compilatore C incluso in \gls{gcc}.
\end{itemize}



 