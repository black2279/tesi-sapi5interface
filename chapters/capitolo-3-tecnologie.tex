\chapter{Tecnologie Utilizzate} %------------------------------ CHAPTER TITLE
\thispagestyle{empty}

\newpage
\section{Balabolka}
Balabolka è un applicazione desktop freeware disponibile per il sistema operativo Microsoft Windows che svolge la funzione di lettore. Essa offre la possibilità di leggere dei testi selezionando una voce a disposizione.\\
Implementa quasi tutti i metodi lato utente dell'interfaccia SAPI~5, supportando in primo luogo la sintesi, la tracciatura delle parole lette e il controllo dei vari parametri della voce.\\
Questa applicazione è stata usata sia in fase di test che in fase di debug.
\subsection*{Vantaggi}
\begin{itemize}
	\item implementa diverse interfacce per tecnologie TTS tra cui SAPI~5;
	\item dispone di un'interfaccia chiara e semplice;
	\item l'applicazione è gratis.
\end{itemize}
\subsection*{Svantaggi}
\begin{itemize}
	\item è disponibile solo la versione a 32-bit.
\end{itemize}

\section{cURL}
cURL è un progetto software che fornisce uno strumento a linea di comando e una libreria per comunicare attraverso vari protocolli di rete.
Nell'attività di stage verrà utilizzato come libreria per comunicare attraverso chiamate HTTP con l'engine MaryTTS.
L'utilizzo di questa libreria è necessario perchè l'engine risiede in un server che risponde attraverso REST API.
\subsection*{Vantaggi}
\begin{itemize}
	\item compatibile con varie piattaforme;
	\item è scritta in C;
	\item open-source;
	\item è gratuita.
\end{itemize}
\subsection*{Svantaggi}
\begin{itemize}
	\item offre delle API non sempre intuitive.
\end{itemize}
\section{C}
Il C è un linguaggio di programmazione nato nel 1972. Segue i paradigmi di tipo strutturato e procedurale. È considerato un linguaggio di programmazione ad "alto livello" ma integra anche caratteristiche vicine al linguaggio Assembly e al linguaggio macchina.\\
Questo linguaggio sarà utilizzato per apportare le modifiche a Speect e all'implementazione delle funzioni di supporto delle interfacce SAPI~5.
\subsection*{Vantaggi}
\begin{itemize}
	\item mirato all'efficienza;
	\item la sua struttura modulare facilita il debug, il testing e la manutenzione;
	\item ha un'elevata portabilità.
\end{itemize}
\subsection*{Svantaggi}
\begin{itemize}
	\item non è orientato agli oggetti;
	\item per gli utenti neofiti la gestione della memoria può essere complicata e alle volte pericolosa.~\cite{c-advantages-disadvantages}
\end{itemize}
\section{C++}
Il C++ è un linguaggio di programmazione orientato agli oggetti, con tipizzazione statica. È stato sviluppato da Bjarne Stroustrup nel 1983 come un miglioramento del linguaggio C.
Offre molti paradigmi di programmazione che vanno da quello procedurale a quello funzionale.
Il linguaggio C++ verrà utilizzato per implementare l'interfaccia SAPI sottoforma di DLL.
\subsection*{Vantaggi}
\begin{itemize}
	\item è compatibile con il linguaggio C;
	\item offre un alto livello di astrazione;
	\item è mirato all'efficienza e alle performance.
\end{itemize}
\subsection*{Svantaggi}
\begin{itemize}
	\item non supporta nativamente i thread;
	\item i puntatori, le variabili globali e le funzioni friend lo rendono un linguaggio poco sicuro.~\cite{c++-advantages-disadvantages}
\end{itemize}
\section{Git}
Git è un software di controllo versione creato da Linus Torvalds nel 2005. Permette di avere un controllo completo della storia di un software e uno sviluppo contemporaneo da parte di più utenti.
Questo software verrà utilizzato per versionare lo sviluppo dell'engine Speect e dell'interfaccia SAPI.
\subsection*{Vantaggi}
\begin{itemize}
	\item è più veloce rispetto ad altri software di versionamento (esempio: SVN);
	\item permette di lavorare in locale;
	\item è gratuito e open source.~\cite{git-advantages-disadvantages}
\end{itemize}
\subsection*{Svantaggi}
\begin{itemize}
	\item non è sempre intuitivo, richiede dello studio;
	\item non presenta un sistema automatico per la numerazione delle versioni;
	\item risoluzione dei conflitti a volte complicata.
\end{itemize}
\section{GitLab}
GitLab è una piattaforma web per la gestione di repository Git scritta da Dmitriy Zaporozhets e Valery Sizov, lanciata nel 2011. Oltre all'organizzazione delle varie repository, essa fornisce la gestione di pagine wiki e da la possibilità di utilizzare un'issue tracking.
La piattaforma verrà utilizzata per ospitare lo sviluppo del componente SAPI per Windows.
\subsection*{Vantaggi}
\begin{itemize}
	\item numero illimitato di progetti privati e collaboratori con account gratuito;
	\item possibilità di aggiungere sistemi di Continuous Integration;
	\item è possibile attribuire differenti ruoli ai collaboratori per gestire i loro permessi.
\end{itemize}
\subsection*{Svantaggi}
\begin{itemize}
	\item non è aperta ad altri sistemi di controllo versione.
\end{itemize}
\section{GitHub}
GitHub è una piattaforma web nata principalmente per la gestione di repository Git, sviluppata da GitHub Inc. e lanciata nel 2008. Come altri servizi di hosting di codice sorgente, offre anche uno strumento di issue tracking e la possibilità di creare pagine wiki. Al suo interno è presente anche un altro componente molto utile chiamato Gist. Il suo comportamento è molto simile ad un altro strumento presente online chiamato Pastebin, usato per condividere codice velocemente, l'unica differenza è la presenza del versionamento tramite Git.
GitHub verrà utilizzato per portare avanti lo sviluppo di Speect e per entrare in possesso della libreria di pos-tagging Hunpos.
\subsection*{Vantaggi}
\begin{itemize}
	\item numero illimitato di repository pubbliche con account gratuito;
\end{itemize}
\subsection*{Svantaggi}
\begin{itemize}
	\item con un account gratuito non è possibile avere repository private;
	\item non è possibile replicare il servizio su altre macchine.
\end{itemize}
\section{MaryTTS}
MaryTTS è un motore di sintesi vocale open-source scritto in Java. E' stato sviluppato inizialmente dalla collaborazione di DFKI e dall'Institute of Phonetics dell'Università di Saarland. La versione che verrà impiegata per condurre il lavoro è stata sviluppata appositamente da Mivoq e prende il nome di FATTS MaryTTS. La versione utilizzata presenta delle differenze rispetto all'originale soprattutto a livello di output messi a disposizione.
Questo engine verrà utilizzato per rispondere alle richieste provenienti dall'interfaccia SAPI.
\subsection*{Vantaggi}
\begin{itemize}
	\item è gratuito e open source;
	\item ha un'alta portabilità perchè è scritto in Java;
	\item l'engine è pensato come un server HTTP;
	\item la versione proposta da Mivoq mette a disposizione delle Web API che rispondono in formato JSON;
	\item supporta molte lingue.
\end{itemize}
\subsection*{Svantaggi}
\begin{itemize}
	\item le API originali rispondono in un formato non sempre adeguato;
	\item la modifica e l'implementazione di nuove features non è sempre semplice.
\end{itemize}
\section{Microsoft Windows}
\section{NSIS}
\section{SAPI~5}
\section{Speect}
\section{TTSApplication}
\section{Visual Studio 2015}
