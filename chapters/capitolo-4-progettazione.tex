\chapter{Progettazione} %------------------------------ CHAPTER TITLE
\thispagestyle{empty}

\newpage
\section{Architettura}
L'architettura ad alto livello che si vuole utilizzare è molto semplice. Essa prevede l'implementazione di un componente SAPI~5 e lo sviluppo di un canale di comunicazione verso un engine TTS.\\
Per facilitare lo svolgimento del progetto si è scelto di utilizzare, in congiunta con il committente, un esempio guida chiamato SALB.\\
SALB è un progetto open source pensato per offrire tramite la specifica SAPI~5 le funzionalità di sintesi vocale ottenute dall'utilizzo congiunto di flite e delle voci basate su modelli HMM elaborate da \gls{hts}.\\
Partendo da questo progetto, andranno analizzate le varie parti che lo compongono e verranno individuate le best practice da introdurre nel progetto di stage.\\
Riassumendo, i vari componenti che verranno utilizzati sono:
\begin{description}
	\item[Interfaccia SAPI~5] interfaccia messa a disposizione dal sistema operativo Microsoft Windows che permette di implementare le funzionalità di sintesi e riconoscimento vocale;
	\item[Implementazione interfaccia SAPI~5] insieme delle funzionalità in grado di permettere la comunicazione tramite la specifica SAPI~5 tra l'engine TTS scelto e la sua destinazione, che potrà essere il sistema operativo o le applicazioni utente; 
	\item[Engine TTS] engine di sintesi vocale messo a disposizione dall'azienda, che può essere MaryTTS o Speect.
\end{description}
	
	
	