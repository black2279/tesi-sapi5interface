\chapter{Progettazione} %------------------------------ CHAPTER TITLE
\thispagestyle{empty}

\newpage
\section{Architettura}
L'architettura ad alto livello che si vuole utilizzare è molto semplice. Essa prevede l'implementazione di un componente SAPI~5 e lo sviluppo di un canale di comunicazione con un engine TTS.
Per facilitare lo svolgimento del progetto si è scelto di utilizzare, in congiunta con il committente, un esempio guida chiamato SALB.
Il progetto SALB implementa l'interfaccia SAPI~5 e al suo interno esegue delle chiamate ad un engine TTS.
Partendo da questo esempio andranno sostituite le varie parti che dovranno subire delle modifiche per poi proseguire fino ad ottenere il prodotto desiderato.
Riassumendo, i vari componenti che verranno utilizzati sono:
\begin{description}
	\item[Interfaccia SAPI~5] interfaccia messa a disposizione dal sistema operativo Microsoft Windows che permette di implementare le funzionalità di sintesi e riconoscimento vocale;
	\item[Implementazione interfaccia SAPI~5] seguendo la specifica SAPI~5 si andranno ad implementare le varie funzionalità di sintesi vocale che potranno essere messe a disposizione attraverso l'interfaccia. L'implementazione fornita dipenderà da un engine TTS.
	\item[Engine TTS] engine di sintesi vocale messo a disposizione dall'azienda. Gli engine di sintesi vocale saranno MaryTTS e Speect.
\end{description}
\newpage

\section{Descrizione dei componenti}
	\subsection{Interfaccia SAPI~5}
	  \begin{figure}[H]
	  	\centering
	  	\includegraphics{images/sapi5-interface.png}
	  	\caption{Interfaccia SAPI~5}
	  \end{figure}
 
L'interfaccia SAPI~5 è stata progettata per facilitare il lavoro agli sviluppatori di applicazioni e di engine TTS. Il suo scopo è quello di standardizzare il modo con cui avviene la comunicazione tra applicazioni ed engine TTS. Uno dei task che viene semplificato, utilizzando questa interfaccia, è la gestione del flusso audio. Questo permette agli sviluppatori focalizzarsi maggiormente nello sviluppo dell'engine e non di come il suo output audio venga gestito.
Le interfacce che verranno utilizzate sono ISpTTSEngine e ISpObjectWithToken.
     \subsubsection{ISpObjectWithToken}
     L'interfaccia ISpObjectWithToken permette di creare e reperire le informazioni associate ad un oggetto. Nella maggior parte dei casi l'oggetto viene associato ad una voce.
     \\\\
     \textbf{Metodi}
     \begin{itemize}
     	\item \texttt{+ SetObjectToken(): HRESULT} imposta le proprietà dell'oggetto passato come parametro.
     	\\\\
     	\textbf{Argomenti}
		\begin{itemize}
			\item \texttt{pToken: ISpObjectToken} puntatore all'oggetto che deve essere impostato.
		\end{itemize}
     	\textbf{Valori di ritorno}
		\begin{itemize}
		 	\item \texttt{S\_OK} se la funzione non ha generato errori;
		 	\item \texttt{E\_POINTER} se il parametro \texttt{pToken} non è valido o è malformato;
		 	\item \texttt{E\_OUTOFMEMORY} se la memoria disponibile non è sufficiente;
		 	\item \texttt{FAILED(hr)} se c'è un errore da ritornare. 
		 \end{itemize}
	    
	     \item \texttt{+ GetObjectToken(): HRESULT} serve per ottenere un oggetto settato in precedenza. L'oggetto verrà restituito come parametro.
	     \\\\
	     \textbf{Argomenti}
		 \begin{itemize}
		     \item \texttt{ppToken: ISpObjectToken} indirizzo dell'oggetto richiesto.
		 \end{itemize}
	     \textbf{Valori di ritorno}
		 \begin{itemize}
		    \item \texttt{S\_OK} se la funzione non ha generato errori;
		    \item \texttt{E\_POINTER} se il parametro \texttt{ppToken} non è valido o è malformato;
		    \item \texttt{E\_OUTOFMEMORY} se la memoria disponibile non è sufficiente;
	     	\item \texttt{FAILED(hr)} se c'è un errore da ritornare. 
	     \end{itemize}
     \end{itemize}
      
     \subsubsection{ISpTTSEngine}
     ISpTTSEngine è l'interfaccia principale della specifica SAPI~5. Essa svolge i due compiti principali della sintesi vocale che sono le chiamate all'engine TTS e la generazione dell'audio in un formato specifico. Inoltre gestisce anche gli eventi generati dalla sintesi vocale.
     \\\\
     \textbf{Metodi}
     \begin{itemize}
     	\item \texttt{+ GetOutputFormat(): HRESULT} ritorna il formato audio previsto dall'engine TTS;
     	\\\\
		\textbf{Argomenti}
		\begin{itemize}
			\item \texttt{pTargetFmtId: GUID} id del formato richiesto come uscita. I valori possibili possono essere di due tipi: \texttt{SPDFID\_Text} per ottenere un formato testuale e \texttt{SPDFID\_WAVEFORMATEX} per un formato audio;
			\item \texttt{pTargetWaveFormatEx: WAVEFORMATEX} se l'identificatore del formato è del tipo \texttt{SPDFID\_WAVEFORMATEX} l'argomento contiene il puntatore alla struttura del formato audio, altrimenti il suo valore è \texttt{NULL};
			\item \texttt{pOutputFormatId: GUID} contiene l'identificatore del formato di uscita che può essere \texttt{SPDFID\_Text} o \texttt{SPDFID\_WAVEFORMATEX}
			\item \texttt{ppCoMemOutputWaveFormatEx: WAVEFORMATEX} contiene la struttura di tipo \texttt{WAVEFORMATEX} del formato audio se l'argomento \texttt{pOutputFormatId} è impostato al valore \texttt{SPDFID\_WAVEFORMATEX} altrimenti è \texttt{NULL}. La struttura verrà allocata tramite \texttt{CoTaskMemAlloc}.
		\end{itemize}
		\item \texttt{+ Speak(): HRESULT} è il metodo che effettua la sintesi vocale trasformando l'input testuale in un formato specifico.
		\\\\
		\textbf{Argomenti}
		\begin{itemize}
			\item \texttt{dwSpeakFlags: DWORD} contiene i valori dei flags che descrivono le caratteristiche dell'input;
			\item \texttt{rguidFormatId: GUID} identificatore del formato di uscita della sintesi vocale. I possibili valori possono essere \texttt{SPDFID\_Text} o \texttt{SPDFID\_WAVEFORMATEX};
			\item \texttt{pWaveFormatEx: WAVEFORMATEX} puntatore alla struttura che descrive il formato d'uscita se il parametro \texttt{rguidFormatId} ha valore \texttt{SPDFID\_WAVEFORMATEX}. L'argomento ha valore \texttt{NULL} se il parametro \texttt{rguidFormatId} ha valore \texttt{SPDFID\_Text};
			\item \texttt{pTextFragList: SPVTEXTFRAG} lista concatenata di \texttt{SPVTEXTFRAG} su cui eseguire la sintesi vocale. Un elemento \texttt{SPVTEXTFRAG} è formato da un frammento di testo decorato da altri atttributi che ne descrivono meglio le caratteristiche;
			\item \texttt{pOutputSite: ISpTTSEngineSite} è il puntatore all'interfaccia \texttt{ISpTTSEngineSite} che viene utilizzato per scrivere l'audio e aggiungere gli eventi SAPI alla coda gestita dall'interfaccia.
		\end{itemize}  
	\end{itemize}

	\subsubsection{Descrizione del funzionamento}
	Per comprendere al meglio il funzionamento dell'interfaccia SAPI~5 verranno descritte le operazioni che tipicamente vengono eseguite.
	Per prima cosa viene effettuata l'inizializzazione dell'engine TTS tramite il metodo \texttt{ISpObjectWithToken::SetObjectToken()}. Questo permette di ottenere un oggetto che può essere utilizzato dall'engine per eseguire la sua inizializzazione.
	Ad esempio attraverso questa operazione è possibile impostare la voce con cui verrà effettuata la sintesi vocale.
	Una volta che l'engine è stato inizializzato è possibile recuperare le sue informazioni tramite il metodo \texttt{ISpObjectWithToken::GetObjectToken()} e di renderle disponibili, se richieste, ad applicazioni o altri componenti.
	Il compito principale di sintesi vocale è affidato al metodo \texttt{ISpTTSEngineSite::Speak()} che si occupa di ricevere l'input testuale, somministrarlo all'engine TTS e scrivere l'output audio in buffer che verrà riprodotto dall'sistema operativo.
	Il formato audio che si vuole scrivere deve essere specificato tramite il metodo \texttt{ISpTTSEngineSite::GetOutputFormat()} in modo che l'acquisizione e la riproduzione avvenga nel modo corretto.\\\\
	\textbf{Eventi}\\
	L'interfaccia SAPI~5 è stata progettata ad eventi e questo comporta che ogni cambiamento di stato sia determinato da un evento.\\
	Un evento SAPI è definito da una struttura chiamata \texttt{SPEVENT} che è composta da i seguenti attributi:
	\begin{itemize}
		\item \texttt{eEventId: WORD} identifica il tipo di evento tramite un enumeratore;
		\item \texttt{elParamType: WORD} definisce tramite un enumeratore il tipo di \texttt{lParam};
		\item \texttt{ulStreamNum: ULONG} identifica a quale stream appartiene l'evento. Nel nostro caso l'engine TTS non deve preoccuparsi di settare questo parametro, perchè sarà compito dell'applicazione;
		\item \texttt{ullAudioOffset: ULONGLONG} rappresenta l'offset dello stream audio espresso in termini di byte. Buona norma è che questo valore coincida con l'inizio di un campione audio;
		\item \texttt{wParam: WPARAM} è un campo generico che può contenere le informazioni associate all'evento;
		\item \texttt{lParam: LPARAM} è un campo generico che può contenere le informazioni associate all'evento.
	\end{itemize}
	Adesso andremo ad analizzare gli eventi che possono essere gestiti dall'interfaccia SAPI in base al loro \texttt{eEventId}: 
	\begin{description}
		\item [] \texttt{SPEI\_TTS\_BOOKMARK} è l'evento associato al raggiungimento di un \texttt{BOOKMARK}.\\
		I suoi campi generici assumono i seguenti significati:
		\begin{itemize}
			\item \texttt{wParam} contiene la conversione della stringa associata al \texttt{BOOKMARK} nel valore numerico;
			\item \texttt{lParam} contiene la stringa associata al \texttt{BOOKMARK}.
		\end{itemize}
		\item [] \texttt{SPEI\_WORD\_BOUNDARY} è l'evento sollevato in corrispondenza dell'inizio di una parola. Esso viene spesso utilizzato per ottenere l'evidenziazione di una parola sintetizzata in un testo.\\
		I suoi campi generici assumono i seguenti significati:
		\begin{itemize}
			\item \texttt{wParam} rappresenta l'offset espresso in caratteri rispetto all'inizio dell'input;
			\item \texttt{lParam} contiene la lunghezza espressa in caratteri della parola che deve essere sintetizzata.
		\end{itemize}
		\item [] \texttt{SPEI\_SENTENCE\_BOUNDARY} è l'evento sollevato in corrispondenza dell'inizio di una frase.\\
		I suoi campi generici assumono i seguenti significati:
		\begin{itemize}
			\item \texttt{wParam} rappresenta l'offset espresso in caratteri rispetto all'inizio dell'input;
			\item \texttt{lParam} contiene la lunghezza espressa in caratteri della frase che deve essere sintetizzata.
		\end{itemize}
		\item [] \texttt{SPEI\_PHONEME} è l'evento sollevato in corrispondenza della presenza di un fonema. Esso può essere utilizzato per effettuare lo spelling fonetico del testo.\\
		I suoi campi generici assumono i seguenti significati:
		\begin{itemize}
			\item \texttt{wParam} è diviso in due \texttt{WORD}, la \texttt{WORD} più significativa contiene la durata in millisecondi del fonema corrente, invece la \texttt{WORD} meno significativa contiene il \texttt{PhoneID} del fonema successivo;
			\item \texttt{lParam} è diviso in due \texttt{WORD}, la \texttt{WORD} più significativa contiene la \texttt{SPVFEATURE} associata al fonema, invece la \texttt{WORD} meno significativa contiene il \texttt{PhoneID} del fonema corrente.
		\end{itemize}
		\item [] \texttt{SPEI\_VISEME} è l'evento sollevato in corrispondenza della presenza di un visema. Gli eventi di questo tipo possono essere utilizzati per pilotare un avatar.\\
		I suoi campi generici assumono i seguenti significati:
		\begin{itemize}
			\item \texttt{wParam} è diviso in due \texttt{WORD}, la \texttt{WORD} più significativa contiene la durata in millisecondi del visema corrente, invece la \texttt{WORD} meno significativa contiene l'identificatore del visema successivo;
			\item \texttt{lParam} è diviso in due \texttt{WORD}, la \texttt{WORD} più significativa contiene la \texttt{SPVFEATURE} associata al visema, invece la \texttt{WORD} meno significativa contiene l'identificatore del visema corrente.
		\end{itemize}
	\end{description}
	L'interfaccia SAPI affida la gestione degli eventi ad una coda. Durante la chiamata al metodo \texttt{ISpTTSEngine::Speak()} è possibile eseguire l'inserimento degli eventi nella coda tramite il metodo \texttt{ISpTTSEngineSite::AddEvents()} mediante l'argomento \texttt{pOutputSite}.
	L'aspetto più importante è dato dalla relazione che esiste tra lo stream audio e gli eventi. Infatti le due entità vengono trattate come se lo stream audio fosse la linea del tempo e gli eventi degli avvenimenti che accadono in determinato momento identificato dal campo \texttt{ullAudioOffset}.
	%immagine linea del tempo con eventi
	\\\\
	\textbf{Azioni}\\
	Lo standard SAPI~5 permette la gestione di azioni che avvengono in tempo reale. Ad esempio è possibile controllare il volume dell'uscita audio o la sua velocità.
	Per fare ciò bisogna utilizzare il metodo \texttt{ISpTTSEngineSite::GetActions()} messo a disposizione dall'argomento \texttt{pOutputSite}.
	L'invocazione del metodo permette di capire quali azioni sono state effettuate dall'applicazione.Esse possono essere di tre tipi:
	\begin{itemize}
		\item \texttt{SPVES\_VOLUME} se il volume è stato cambiato;
		\item \texttt{SPVES\_RATE} se la velocità è stata cambiata;
		\item \texttt{SPVES\_SKIP} se è stata espressa l'intenzione di spostarsi ad un certo punto del testo;
		\item \texttt{SPVES\_ABORT} se la sintesi è stata interrotta;
		\item \texttt{SPVES\_CONTINUE} se nessuna delle precedenti azioni è stata eseguita;
	\end{itemize}
	Nei i primi tre casi l'interfaccia SAPI mette a disposizione dei metodi per recuperare i nuovi valori e sono rispettivamente:
	\begin{itemize}
		\item \texttt{ISpTTSEngineSite::GetVolume(): HRESULT} per recuperare il nuovo valore del volume;
		\item \texttt{ISpTTSEngineSite::GetRate(): HRESULT} per recuperare il nuovo valore della velocità;
		\item \texttt{ISpTTSEngineSite::GetSkipInfo(): HRESULT} per recuperare il tipo e il numero di unità da saltare in avanti o indietro rispetto al punto in cui la sintesi è arrivata. Per completare l'operazione c'è il bisogno di invocare il metodo \texttt{ISpTTSEngineSite::CompleteSkip()} che verifica se è possibile portare a termine l'azione;
	\end{itemize}	
	Un altro parametro della voce su cui è possibile agire è la tonalità. In questo caso la regolazione non avviene come per la velocità o il volume, ma si deve agire sulla proprietà \texttt{PitchAdj} associata agli elementi della lista \texttt{pTextFragList}.
	\subsection{Implementazione Interfaccia SAPI~5 per MaryTTS}
		%\begin{figure}[H]
		%	\centering
		%	\includegraphics{images/sapi5-interface.png}
		%	\caption{Interfaccia SAPI~5}
		%\end{figure}
	Questa implementazione prevede l'utilizzo dell'engine MaryTTS fornito da Mivoq. L'engine non è disponibile direttamente sottoforma di libreria, ma è integrato in una piattaforma chiamata FA-TTS. Questa piattaforma è scritta in Java ed è essenzialmente un server che risponde alle richieste HTTP in modo REST e nel formato JSON.
	Per rendere possibile la comunicazione tra l'implementazione dell'interfaccia SAPI e MaryTTS si è scelto di utilizzare la libreria cURL per eseguire le richieste HTTP al server.
	Per gestire in maniera ottimale l'output del server, invece è stata scelta la libreria Jansson che permette di eseguire il parsing degli oggetti JSON.
	I compiti a cui dovrà adempire questa implementazione saranno:
	\begin{itemize}
		\item richiedere a MaryTTS di eseguire la sintesi vocale dell'input richiesto;
		\item gestire l'output audio e renderlo disponibile al sistema operativo.
		\item eseguire la sintesi vocale con determinati parametri della voce.
	\end{itemize}
	La forma binaria che assumerà questa implementazione non sarà un'eseguibile ma una libreria dinamica. Questo permetterà al sistema operativo e a tutte le altre applicazioni di poterla utilizzare senza problemi attraverso lo standard SAPI~5.
	\subsubsection{FATTS}
	FATTS è la classe che svolge i compiti principali: implementa le interfacce SAPI~5 e dialoga con l'engine MaryTTS tramite la libreria cURL attraverso chiamate HTTP. La parte relativa a SAPI~5 è gestita internamente attraverso i suoi metodi privati, invece la comunicazione con il server è affidata all'entità \texttt{fatts\_client}. FATTS si occupa quindi di gestire gli eventi, le azioni provenienti dalle applicazioni, il formato dell'output e la preparazione dell'input da inviare al server.\\    
	\textbf{Implementa:}
	\begin{itemize}
		\item \texttt{ISpTTSEngine};
		\item \texttt{ISpTTSObjectWithToken};
	\end{itemize}
	\textbf{Eredita da:}
	\begin{itemize}
		\item \texttt{CComObjectRootEx};
		\item \texttt{CComCoClass};
	\end{itemize}
	\textbf{Attributi}
		\begin{itemize}
			\item \texttt{- voiceToken: CComPtr<ISpObjectToken>} rappresenta l'oggetto che mantiene le informazioni relative alla voce;
			\item \texttt{- voiceProperties: FragmentPropertiesPtr} riferimento ad una mappa che contiene le informazioni relative alla voce in modo centralizzato;
			\item \texttt{- actionAborted: boolean} flag che viene utilizzato per interrompere la sintesi vocale;
			\item \texttt{- actionSkipSentences: int} indica il numero di frasi che la sintesi vocale deve saltare durante l'elaborazione;   
		\end{itemize}
	\textbf{Metodi}
	\begin{itemize}
		\item \texttt{+ FinalConstruct(): HRESULT} override del metodo ereditato dalla classe \texttt{CComObjectRootEx} che serve per eseguire le inizializzazioni necessarie dell'oggetto;
		\item \texttt{+ FinalRelease(): void} override del metodo ereditato dalla classe \texttt{CComObjectRootEx} che serve per eseguire la pulizia dell'oggetto prima di distruggerlo;
		\item \texttt{+ SetObjectToken(): HRESULT} metodo che serve per recuperare le informazioni relative alla voce scritte nel registro di Windows. Questo metodo andrà ad impostare l'attributo \texttt{voiceToken}.\\\\
		\textbf{Argomenti}
		\begin{itemize}
			\item \texttt{pToken: ISpObjectToken} è il riferimento al token della voce contenuto nel registro di Windows. 
		\end{itemize}
		\item \texttt{+ GetObjectToken(): HRESULT} metodo che serve per rendere disponibile l'oggetto \texttt{voiceToken} all'esterno;
		\begin{itemize}
			\item \texttt{ppToken: ISpObjectToken} argomento utilizzato per ritornare il l'oggetto impostato tramite il metodo \texttt{SetObjectToken()};
		\end{itemize}
		\item \texttt{Speak(): HRESULT} metodo che serve per effettuare la sintesi vocale attraverso l'engine MaryTTS. Aggiunge gli eventi alla coda e gestisce l'output dell'engine mediante \texttt{pOutputSite}.\\\\
		\textbf{Argomenti}
		\begin{itemize}
			\item \texttt{dwSpeakFlags: DWORD} contiene i valori dei flags che descrivono le caratteristiche dell'input;
			\item \texttt{rguidFormatId: GUID} identificatore del formato di uscita della sintesi vocale. I possibili valori possono essere \texttt{SPDFID\_Text} o \texttt{SPDFID\_WAVEFORMATEX};
			\item \texttt{pWaveFormatEx: WAVEFORMATEX} puntatore alla struttura che descrive il formato d'uscita se il parametro \texttt{rguidFormatId} ha valore \texttt{SPDFID\_WAVEFORMATEX}. L'argomento ha valore \texttt{NULL} se il parametro \texttt{rguidFormatId} ha valore \texttt{SPDFID\_Text};
			\item \texttt{pTextFragList: SPVTEXTFRAG} lista concatenata di \texttt{SPVTEXTFRAG} su cui eseguire la sintesi vocale. Un elemento \texttt{SPVTEXTFRAG} è formato da un frammento di testo decorato da altri atttributi che ne descrivono meglio le caratteristiche;
			\item \texttt{pOutputSite: ISpTTSEngineSite} è il puntatore all'interfaccia \texttt{ISpTTSEngineSite} che viene utilizzato per scrivere l'audio e aggiungere gli eventi SAPI alla coda.
		\end{itemize}
		\item \texttt{+ GetOutputFormat(): HRESULT} ritorna il formato audio previsto dall'engine MaryTTS;
		\\\\
		\textbf{Argomenti}
		\begin{itemize}
			\item \texttt{pTargetFmtId: GUID} id del formato richiesto come uscita. I valori possibili possono essere di due tipi: \texttt{SPDFID\_Text} per ottenere un formato testuale e \texttt{SPDFID\_WAVEFORMATEX} per un formato audio;
			\item \texttt{pTargetWaveFormatEx: WAVEFORMATEX} se l'identificatore del formato è del tipo \texttt{SPDFID\_WAVEFORMATEX} l'argomento contiene il puntatore alla struttura del formato audio, altrimenti il suo valore è \texttt{NULL};
			\item \texttt{pOutputFormatId: GUID} contiene l'identificatore del formato di uscita che può essere \texttt{SPDFID\_Text} o \texttt{SPDFID\_WAVEFORMATEX}
			\item \texttt{ppCoMemOutputWaveFormatEx: WAVEFORMATEX} contiene la struttura di tipo \texttt{WAVEFORMATEX} del formato audio se l'argomento \texttt{pOutputFormatId} è impostato al valore \texttt{SPDFID\_WAVEFORMATEX} altrimenti è \texttt{NULL}. La struttura verrà allocata tramite \texttt{CoTaskMemAlloc}.
		\end{itemize}
		\item \texttt{- ResetActions(): void} metodo che ha il compito di ripristinare l'evoluzione delle azioni compiute allo stato iniziale comportando l'interruzione della sintesi vocale.
		\item \texttt{- HandleActions(): void} metodo che permette di gestire le azioni provenienti dall'applicazione, come la regolazione del volume e della velocità di riproduzione.
		\\\\
		\textbf{Argomenti}
		\begin{itemize}
			\item \texttt{site: ISpTTSEngineSite} riferimento all'interfaccia \texttt{ISpTTSEngineSite} per permettere il recupero delle azioni svolte dall'applicazione;
			\item \texttt{utterance: json\_t} oggetto che rappresenta l'utterance corrente e  viene modificato in base alle azioni che vengono compiute. Ad esempio ad una variazione del volume, l'oggetto in questione viene modificato per permettere la medesima regolazione all'interno dell'engine MaryTTS;
		\end{itemize}
		\item \texttt{- HandleEventInterests(): void} metodo che viene utilizzato all'interno della classe \texttt{FATTS} per capire la tipologia di eventi che vengono sollevati. Questo metodo è utile per variare il comportamento della classe in base agli eventi che si presentano;\\\\
		\textbf{Argomenti}
		\begin{itemize}
			\item \texttt{site: ISpTTSEngineSite} riferimento all'interfaccia \texttt{ISpTTSEngineSite} per permettere la chiamata \texttt{ISpTTSEngineSite::GetEventInterest()} in modo da recuperare la tipologia di eventi sollevata.
		\end{itemize}
		\item \texttt{AdjustProperties(): FragmentPropertiesPtr} metodo che serve per aggiornare le proprietà associate alla voce. L'aggiornamento viene effettuato solamente se sono presenti nuovi valori, altrimenti le proprietà restano invariate.
		\\\\
		\textbf{Argomenti}
		\begin{itemize}
			\item \texttt{state: SPVSTATE} rappresenta l'insieme delle proprietà associate ad un frammento del testo;
			\item \texttt{props: FragmentPropertiesPtr} mappa che contiene le proprietà della voce;
			\item \texttt{utterance: json\_t} oggetto che viene utilizzato da MaryTTS per modificare le caratteristiche della voce durante la sintesi vocale. In particolare, in questo metodo viene utilizzato per modificare la tonalità della voce.
		\end{itemize} 		
	\end{itemize}
	\subsubsection{fatts\_client}
	\texttt{fatts\_client} è il modulo che comunica direttamente con l'engine MaryTTS attraverso la libreria cURL. L'interazione tra il modulo e MaryTTS avviene mediante richieste HTTP. Invece, per gestire l'output del server in formato JSON si è scelto di utilizzare la libreria Jansson. 
	Il modulo si compone di pochi metodi quali l'inizializzazione, la pulizia e la richiesta al server.
	Il metodo che effettua la richiesta al server sfrutta il meccanismo delle callback. Infatti gli argomenti del metodo non accettano solo oggetti ma anche funzioni. Questo implica che le callback potranno essere definite all'esterno e assumere vari comportamenti.
	\\\\
	\textbf{Metodi}
	\begin{itemize}
		\item \texttt{fatts\_init(): int} metodo che esegue l'inizializzazione della libreria cURL;
		\item \texttt{fatts\_cleanup(): void} metodo che ha il compito di eseguire la pulizia degli oggetti utilizzati dal client e dalla libreria cURL;
		\item \texttt{fatts\_client\_say\_complete(): int} metodo che esegue la richiesta completa al server. Si occupa di recuperare attraverso l'engine MaryTTS i fonemi, i visemi, le informazioni legate alle parole sintetizzate e il flusso audio da riprodurre.
		Ogni operazione viene effettuata mediante una callback passata come argomento.
		\\\\
		\textbf{Argomenti}
		\begin{itemize}
			\item \texttt{conn: fatts\_client\_s} oggetto che contiene le informazioni relative alla connessione con il server;
			\item \texttt{req : fatts\_request\_s} oggetto che raccoglie tutti i parametri che servono ad effettuare la richiesta al server. I parametri in questione contengono la configurazione della voce, il formato dell'input, l'input testuale e il tipo dell'output desiderato;
			\item \texttt{event\_cb: void} callback utilizzata per aggiungere gli eventi legati ai fonemi e ai visemi alla coda degli eventi tenuta dallo standard SAPI;
			\item \texttt{event\_cb\_data: void} oggetto utilizzato per poter permettere il passaggio dei dati utili alla callback \texttt{event\_cb};
			\item \texttt{word\_event\_cb: void} callback utilizzata per costruire gli eventi legati alle parole da sintetizzare.
			\item \texttt{word\_event\_cb\_data: void} oggetto di supporto utilizzato per rendere possibile il funzionamento della callback \texttt{event\_word\_cb}.
			\item \texttt{write\_callback: size\_t} callback utilizzata per trasferire l'audio proveniente dal server nel buffer che verrà riprodotto.
			\item \texttt{stream: void} rappresenta l'oggetto che verrà utilizzato dalla callback \texttt{write\_callback} per scrivere lo stream audio nel buffer messo a disposizione dallo standard SAPI.
		\end{itemize}
	\end{itemize}
	\subsubsection{Descrizione del funzionamento}
	Adesso andremo a descrivere il flusso delle operazioni effettuate dall'implementazione con un esempio.
	Gli input che l'utente può scegliere tipicamente sono:
	\begin{itemize}
		\item la voce;
		\item il testo su cui effettuare la sintesi;
		\item il volume;
		\item la velocità;
		\item la tonalità della voce;
	\end{itemize}
	Una volta scelti questi parametri è possibile iniziare la sintesi vocale. Ricordiamo che la voce e tutti i suoi parametri sono rappresentati da un token presente nel registro di Windows.
	Quindi nel momento della selezione della voce verranno invocati i metodi \texttt{FATTS:SetObjectToken()} e \texttt{FATTS:GetObjectToken()} per rendere disponibile sia all'applicazione sia all'engine MaryTTS le informazioni relative alla voce.
	Dopo questa operazione viene invocato il metodo \texttt{FATTS::Speak()} che esegue la sintesi vocale.
	All'interno del metodo \texttt{FATTS::Speak()} in un primo momento viene analizzato il testo proveniente dall'applicazione, vengono estratte le proprietà come la tonalità della voce e poi viene ricomposto per essere spedito al server.
	Prima di effettuare la richiesta, viene invocato il metodo \texttt{HandleActions()} che permette di recuperare i parametri mancanti come il volume e la velocità e nel medesimo frangente vengono costruiti gli oggetti utili alle callback che avranno il compito di costruire ed aggiungere gli eventi alla coda.
	A questo punto il metodo \texttt{fatts\_client\_say\_complete()} è pronto per essere chiamato ed eseguire la richiesta al server.
	Il metodo sfruttando la libreria cURL e Jansson elabora l'output del server in formato JSON, esegue la costruzione degli eventi e invia lo stream audio alla riproduzione. 
	
	
	