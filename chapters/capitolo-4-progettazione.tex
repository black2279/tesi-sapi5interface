\chapter{Progettazione} %------------------------------ CHAPTER TITLE
\thispagestyle{empty}

\newpage
\section{Architettura}
L'architettura ad alto livello che si vuole utilizzare è molto semplice. Essa prevede l'implementazione di un componente SAPI~5 e lo sviluppo di un canale di comunicazione con un engine TTS.
Per facilitare lo svolgimento del progetto si è scelto di utilizzare, in congiunta con il committente, un esempio guida chiamato SALB.
Il progetto SALB implementa l'interfaccia SAPI~5 e al suo interno esegue delle chiamate ad un engine TTS.
Partendo da questo esempio andranno sostituite le varie parti che dovranno subire delle modifiche per poi proseguire fino ad ottenere il prodotto desiderato.
Riassumendo, i vari componenti che verranno utilizzati sono:
\begin{description}
	\item[Interfaccia SAPI~5] interfaccia messa a disposizione dal sistema operativo Microsoft Windows che permette di implementare le funzionalità di sintesi e riconoscimento vocale;
	\item[Implementazione interfaccia SAPI~5] seguendo la specifica SAPI~5 si andranno ad implementare le varie funzionalità di sintesi vocale che potranno essere messe a disposizione attraverso l'interfaccia. L'implementazione fornita dipenderà da un engine TTS.
	\item[Engine TTS] engine di sintesi vocale messo a disposizione dall'azienda. Gli engine di sintesi vocale saranno MaryTTS e Speect.
\end{description}
	
	
	