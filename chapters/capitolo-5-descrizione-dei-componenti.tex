\chapter{Descrizione dei componenti}
\label{chap:descrizione-dei-componenti}
Dopo aver progettato i componenti software e aver individuato la posizione più consona all'interno del sistema si procede con l'implementazione.\\
Di seguito verranno descritte in dettaglio le implementazioni sviluppate.\\
Per ogni componente sono stati individuati i metodi più significativi da illustrare e verrà fornito un esempio tipico di funzionamento in modo da chiarire il ruolo delle varie parti in gioco.\\
Per ottenere la massima chiarezza si affronterà una descrizione dal generale al particolare. 
 
\thispagestyle{empty}

\newpage
\section{Interfaccia SAPI~5}
\begin{figure}[H]
	\centering
	\includegraphics{images/sapi5-interface.png}
	\caption{Interfaccia SAPI~5}
\end{figure}

L'interfaccia SAPI~5 è stata progettata per standardizzare la comunicazione tra applicazioni ed engine TTS, in modo da facilitare il lavoro agli sviluppatori. Uno dei vantaggi che offre questa interfaccia è che gli sviluppatori di engine TTS non si devono più preoccupare della gestione dell'output audio.\\
Le interfacce che verranno utilizzate durante l'implementazione sono \textbf{ISpTTSEngine} e \textbf{ISpObjectWithToken}.
\subsection{ISpObjectWithToken}
L'interfaccia ISpObjectWithToken permette di accedere ad un oggetto associato ad un \gls{tokeng}. Nel nostro caso un oggetto di questo tipo è la voce utilizzata da un engine TTS. Secondo la specifica SAPI~5 le voci risiedono nel registro di Windows e rappresentano il punto di inizializzazione per i sistemi di sintesi vocale. 
\\\\
\textbf{Metodi}
\begin{itemize}
	\item \texttt{+ SetObjectToken(): HRESULT} associa un oggetto ad un token specifico e permette di recuperare tutte le sue informazioni.
	\\\\
	\textbf{Argomenti}
	\begin{itemize}
		\item \texttt{pToken: ISpObjectToken} puntatore al token che deve essere associato all'oggetto.
	\end{itemize}
	\textbf{Valori di ritorno}
	\begin{itemize}
		\item \texttt{S\_OK} se la funzione non ha generato errori;
		\item \texttt{E\_POINTER} se il parametro \texttt{pToken} non è valido o è malformato;
		\item \texttt{E\_OUTOFMEMORY} se la memoria disponibile non è sufficiente;
		\item \texttt{FAILED(hr)} se c'è un errore da ritornare. 
	\end{itemize}
	
	\item \texttt{+ GetObjectToken(): HRESULT} restituisce attraverso il parametro \texttt{ppToken} l'oggetto associato ad un token.
	\\\\
	\textbf{Argomenti}
	\begin{itemize}
		\item \texttt{ppToken: ISpObjectToken} indirizzo dell'oggetto associato al token che si vuole restituire.
	\end{itemize}
	\textbf{Valori di ritorno}
	\begin{itemize}
		\item \texttt{S\_OK} se la funzione non ha generato errori;
		\item \texttt{E\_POINTER} se il parametro \texttt{ppToken} non è valido o è malformato;
		\item \texttt{E\_OUTOFMEMORY} se la memoria disponibile non è sufficiente;
		\item \texttt{FAILED(hr)} se c'è un errore da ritornare. 
	\end{itemize}
\end{itemize}

\subsection{ISpTTSEngine}
ISpTTSEngine è l'interfaccia principale della specifica SAPI~5 che viene impiegata durante il processo di sintesi vocale.\\
Si preoccupa di eseguire le chiamate all'engine TTS e di trattare l'output audio secondo un formato specifico. Inoltre, ha il compito di gestire gli eventi SAPI che vengono sollevati durante la sintesi vocale.
\\\\
\textbf{Metodi}
\begin{itemize}
	\item \texttt{+ GetOutputFormat(): HRESULT} ritorna il formato audio richiesto. Se l'engine TTS non riesce a soddisfare la richiesta, ha il compito di ritornare un formato supportato molto simile a quello richiesto.
	\\\\
	\textbf{Argomenti}
	\begin{itemize}
		\item \texttt{pTargetFmtId: GUID} id del formato richiesto come uscita. I valori possibili possono essere di due tipi: \texttt{SPDFID\_Text} per ottenere un formato testuale e \texttt{SPDFID\_WAVEFORMATEX} per un formato audio;
		\item \texttt{pTargetWaveFormatEx: WAVEFORMATEX} se l'identificatore del formato è del tipo \texttt{SPDFID\_WAVEFORMATEX} l'argomento contiene il puntatore alla struttura del formato audio, altrimenti il suo valore è \texttt{NULL};
		\item \texttt{pOutputFormatId: GUID} contiene l'identificatore del formato di uscita che può essere \texttt{SPDFID\_Text} o \texttt{SPDFID\_WAVEFORMATEX}
		\item \texttt{ppCoMemOutputWaveFormatEx: WAVEFORMATEX} se l'argomento\\ \texttt{pOutputFormatId} è impostato al valore \texttt{SPDFID\_WAVEFORMATEX} contiene la struttura di tipo \texttt{WAVEFORMATEX} del formato audio altrimenti è \texttt{NULL}. La struttura verrà allocata tramite \texttt{CoTaskMemAlloc}.
	\end{itemize}
	\item \texttt{+ Speak(): HRESULT} metodo che si occupa di trasformare l'input testuale in un formato specifico, sfruttando un engine TTS. 
	\\\\
	\textbf{Argomenti}
	\begin{itemize}
		\item \texttt{dwSpeakFlags: DWORD} contiene i valori dei flags che descrivono le caratteristiche dell'input;
		\item \texttt{rguidFormatId: GUID} identificatore del formato di uscita della sintesi vocale. I possibili valori possono essere \texttt{SPDFID\_Text} o\\ \texttt{SPDFID\_WAVEFORMATEX};
		\item \texttt{pWaveFormatEx: WAVEFORMATEX} puntatore alla struttura che descrive il formato d'uscita se il parametro \texttt{rguidFormatId} ha valore\\ \texttt{SPDFID\_WAVEFORMATEX}. L'argomento ha valore \texttt{NULL} se il parametro \texttt{rguidFormatId} ha valore \texttt{SPDFID\_Text};
		\item \texttt{pTextFragList: SPVTEXTFRAG} lista concatenata di \texttt{SPVTEXTFRAG} usata come input per la sintesi vocale. Un elemento \texttt{SPVTEXTFRAG} è formato da un frammento di testo decorato da altri attributi che ne descrivono meglio le caratteristiche;
		\item \texttt{pOutputSite: ISpTTSEngineSite} puntatore all'interfaccia\\ \texttt{ISpTTSEngineSite} che viene utilizzato per scrivere l'audio nel buffer dedicato e aggiungere gli eventi SAPI alla coda.
	\end{itemize}  
\end{itemize}

\subsection{Descrizione del funzionamento}
Per comprendere al meglio il funzionamento dell'interfaccia SAPI~5 verranno descritte le operazioni che tipicamente vengono eseguite.\\
Per prima cosa viene effettuata l'inizializzazione dell'engine TTS tramite il metodo \texttt{ISpObjectWithToken::SetObjectToken()}. Questo permette di ottenere un oggetto che può essere utilizzato dall'engine per eseguire la sua inizializzazione.
Ad esempio attraverso questa operazione è possibile impostare la voce con cui verrà effettuata la sintesi vocale.\\
Una volta che l'engine è stato inizializzato è possibile recuperare le sue informazioni tramite il metodo \texttt{ISpObjectWithToken::GetObjectToken()} e di renderle disponibili, se richieste, ad applicazioni o altri componenti.\\
La sintesi vocale è affidata al metodo \texttt{ISpTTSEngine::Speak()} che si occupa di ricevere l'input testuale, somministrarlo all'engine TTS e scrivere l'output audio in buffer che verrà successivamente riprodotto dal sistema operativo.\\
Il formato audio che si vuole scrivere deve essere specificato tramite il metodo \texttt{ISpTTSEngine::GetOutputFormat()} in modo che l'acquisizione e la riproduzione avvenga nel modo corretto.\\\\
\textbf{Eventi}\\
L'interfaccia SAPI~5 è stata progettata ad eventi e questo comporta che ogni cambiamento di stato sia determinato da un evento.\\
Un evento SAPI è definito da una struttura chiamata \texttt{SPEVENT} che è composta da i seguenti attributi:
\begin{itemize}
	\item \texttt{eEventId: WORD} identifica il tipo di evento tramite un enumeratore;
	\item \texttt{elParamType: WORD} definisce tramite un enumeratore il tipo di \texttt{lParam};
	\item \texttt{ulStreamNum: ULONG} identifica a quale stream appartiene l'evento. Nel nostro caso l'engine TTS non deve preoccuparsi di settare questo parametro, perché sarà compito dell'applicazione;
	\item \texttt{ullAudioOffset: ULONGLONG} rappresenta l'offset dello stream audio espresso in termini di byte. Buona norma è che questo valore coincida con l'inizio di un campione audio;
	\item \texttt{wParam: WPARAM} è un campo generico che può contenere le informazioni associate all'evento;
	\item \texttt{lParam: LPARAM} è un campo generico che può contenere le informazioni associate all'evento.
\end{itemize}
Adesso andremo ad analizzare gli eventi che possono essere gestiti dall'interfaccia SAPI in base al loro \texttt{eEventId}: 
\begin{description}
	\item [] \texttt{SPEI\_TTS\_BOOKMARK} è l'evento associato al raggiungimento di un \texttt{BOOKMARK}.\\
	I suoi campi generici assumono i seguenti significati:
	\begin{itemize}
		\item \texttt{wParam} contiene la conversione della stringa associata al \texttt{BOOKMARK} nel valore numerico;
		\item \texttt{lParam} contiene la stringa associata al \texttt{BOOKMARK}.
	\end{itemize}
	\item [] \texttt{SPEI\_WORD\_BOUNDARY} è l'evento sollevato in corrispondenza dell'inizio di una parola. Molto spesso viene utilizzato dalle applicazioni per evidenziare le parole sintetizzate in tempo reale.\\
	I suoi campi generici assumono i seguenti significati:
	\begin{itemize}
		\item \texttt{wParam} rappresenta l'offset espresso in caratteri rispetto all'inizio dell'input;
		\item \texttt{lParam} contiene la lunghezza espressa in caratteri della parola che deve essere sintetizzata.
	\end{itemize}
	\item [] \texttt{SPEI\_SENTENCE\_BOUNDARY} è l'evento sollevato in corrispondenza dell'inizio di una frase.\\
	I suoi campi generici assumono i seguenti significati:
	\begin{itemize}
		\item \texttt{wParam} rappresenta l'offset espresso in caratteri rispetto all'inizio dell'input;
		\item \texttt{lParam} contiene la lunghezza espressa in caratteri della frase che deve essere sintetizzata.
	\end{itemize}
	\item [] \texttt{SPEI\_PHONEME} è l'evento sollevato in corrispondenza della presenza di un \gls{fonemag}. Esso può essere utilizzato per effettuare lo spelling fonetico del testo.\\
	I suoi campi generici assumono i seguenti significati:
	\begin{itemize}
		\item \texttt{wParam} è diviso in due \texttt{WORD}, la \texttt{WORD} più significativa contiene la durata in millisecondi del fonema corrente, invece la \texttt{WORD} meno significativa contiene il \texttt{PhoneID} del fonema successivo;
		\item \texttt{lParam} è diviso in due \texttt{WORD}, la \texttt{WORD} più significativa contiene la \texttt{SPVFEATURE} associata al fonema, invece la \texttt{WORD} meno significativa contiene il \texttt{PhoneID} del fonema corrente.
	\end{itemize}
	\item [] \texttt{SPEI\_VISEME} è l'evento sollevato in corrispondenza della presenza di un \gls{visemag}. Gli eventi di questo tipo possono essere utilizzati per pilotare un avatar.\\
	I suoi campi generici assumono i seguenti significati:                                                       
	\begin{itemize}                                                                                              
		\item \texttt{wParam} è diviso in due \texttt{WORD}, la \texttt{WORD} più significativa contiene la durata in millisecondi del visema corrente, invece la \texttt{WORD} meno significativa contiene l'identificatore del visema successivo;
		\item \texttt{lParam} è diviso in due \texttt{WORD}, la \texttt{WORD} più significativa contiene la \texttt{SPVFEATURE} associata al visema, invece la \texttt{WORD} meno significativa contiene l'identificatore del visema corrente.
	\end{itemize}                                                                                                
\end{description}                                                                                             
L'interfaccia SAPI utilizza una coda per gestire gli eventi.\\                                                
L'operazione di inserimento degli eventi può essere effettuata tramite il metodo \texttt{ISpTTSEngineSite::AddEvents()}, durante la chiamata \\ \texttt{ISpTTSEngine::Speak()}.\\
Nel momento in cui è atto la sintesi vocale, il lancio corretto degli eventi viene garantito dal campo \texttt{ullAudioOffset} che si preoccupa di associare gli eventi allo stream audio.
\\\\
\textbf{Azioni}\\
Lo standard SAPI~5 permette di gestire le azioni che avvengono in tempo reale. Ad esempio è possibile controllare il volume dell'uscita audio o la sua velocità.
Per fare ciò bisogna utilizzare il metodo \texttt{ISpTTSEngineSite::GetActions()} messo a disposizione dall'argomento \texttt{pOutputSite}.
L'invocazione del metodo permette di capire quali azioni sono state effettuate dall'applicazione.\\
Le azioni che possono essere lanciate sono:
\begin{itemize}
	\item \texttt{SPVES\_VOLUME} se il volume è stato cambiato;
	\item \texttt{SPVES\_RATE} se la velocità è stata cambiata;
	\item \texttt{SPVES\_SKIP} se è stata espressa l'intenzione di spostarsi ad un certo punto del testo;
	\item \texttt{SPVES\_ABORT} se la sintesi è stata interrotta;
	\item \texttt{SPVES\_CONTINUE} se nessuna delle precedenti azioni è stata eseguita;
\end{itemize}
Nei i primi tre casi l'interfaccia SAPI mette a disposizione dei metodi per recuperare i nuovi valori e sono rispettivamente:
\begin{itemize}
	\item \texttt{ISpTTSEngineSite::GetVolume(): HRESULT} per recuperare il nuovo valore del volume;
	\item \texttt{ISpTTSEngineSite::GetRate(): HRESULT} per recuperare il nuovo valore della velocità;
	\item \texttt{ISpTTSEngineSite::GetSkipInfo(): HRESULT} per recuperare il tipo e il numero di unità da saltare in avanti o indietro rispetto al punto in cui la sintesi è arrivata. Per completare l'operazione c'è il bisogno di invocare il metodo \texttt{ISpTTSEngineSite::CompleteSkip()} che verifica se è possibile portare a termine l'azione;
\end{itemize}	
Un altro parametro della voce su cui è possibile agire è la tonalità. In questo caso la regolazione non avviene come per la velocità o il volume, ma si deve agire sulla proprietà \texttt{PitchAdj} associata agli elementi della lista \texttt{pTextFragList}.

\newpage
\section{Implementazione Interfaccia SAPI~5 per MaryTTS}
L'implementazione prevede l'utilizzo dell'engine MaryTTS fornito da Mivoq. Il sistema di sintesi vocale è scritto in Java e non fornisce una libreria per affrontare lo sviluppo. MaryTTS però dispone di un'architettura client-server che ha il vantaggio di poter essere interrogata attraverso richieste HTTP. Nella versione fornita da Mivoq, ovvero FATTS, l'engine è in grado di rispondere in modo \gls{restfulg} e presentare il suo output in formato JSON.\\ 
Il componente che si andrà a sviluppare farà uso di due librerie:
\begin{itemize}
	\item \textbf{cURL} per effettuare le richieste HTTP;
	\item \textbf{\gls{janssong}} per gestire l'output del server in formato JSON;
\end{itemize}
I compiti a cui dovrà adempire questa implementazione saranno:
\begin{itemize}
	\item richiedere la sintesi vocale dell'input richiesto a MaryTTS;
	\item gestire l'output audio e renderlo disponibile al sistema operativo.
	\item eseguire la sintesi vocale con determinati parametri della voce.
\end{itemize}
La forma binaria che assumerà questa implementazione non sarà un'eseguibile ma una libreria dinamica. Questo permetterà al sistema operativo e a tutte le altre applicazioni di poterla utilizzare senza problemi attraverso lo standard SAPI~5.
\subsection{FATTS}
FATTS è la classe che svolge i compiti principali: implementa le interfacce SAPI~5 e dialoga con l'engine MaryTTS tramite la libreria cURL attraverso chiamate HTTP. La parte relativa a SAPI~5 è gestita internamente attraverso i suoi metodi privati, invece la comunicazione con il server è affidata all'entità \texttt{fatts\_client}.\\
FATTS si occupa quindi di gestire gli eventi, le azioni provenienti dalle applicazioni, il formato dell'output e la preparazione dell'input da inviare al server.\\\\    
\textbf{Implementa:}
\begin{itemize}
	\item \texttt{ISpTTSEngine};
	\item \texttt{ISpTTSObjectWithToken};
\end{itemize}
\textbf{Eredita da:}
\begin{itemize}
	\item \texttt{CComObjectRootEx};
	\item \texttt{CComCoClass};
\end{itemize}
\textbf{Attributi}
\begin{itemize}
	\item \texttt{- voiceToken: CComPtr<ISpObjectToken>} rappresenta l'oggetto associato al token che rappresenta la voce;
	\item \texttt{- voiceProperties: FragmentPropertiesPtr} mappa che contiene le proprietà relative alla voce;
	\item \texttt{- actionAborted: boolean} flag che viene utilizzato per interrompere la sintesi vocale;
	\item \texttt{- actionSkipSentences: int} indica il numero di frasi che la sintesi vocale deve saltare durante l'elaborazione;   
\end{itemize}
\textbf{Metodi}
\begin{itemize}
	\item \texttt{+ FATTS()} è costruttore della classe che serve ad inizializzare le proprietà legate alla voce;
	\item \texttt{+ FinalConstruct(): HRESULT} override del metodo ereditato dalla classe \texttt{CComObjectRootEx} che serve per eseguire le inizializzazioni necessarie dell'oggetto;
	\item \texttt{+ FinalRelease(): void} override del metodo ereditato dalla classe\\ \texttt{CComObjectRootEx} che serve per eseguire la pulizia dell'oggetto prima di distruggerlo;
	\item \texttt{+ SetObjectToken(): HRESULT} metodo che serve per recuperare le informazioni relative alla voce scritte nel registro di Windows. Questo metodo andrà ad impostare l'attributo \texttt{voiceToken}.\\\\
	\textbf{Argomenti}
	\begin{itemize}
		\item \texttt{pToken: ISpObjectToken} è il riferimento al token della voce contenuto nel registro di Windows. 
	\end{itemize}
	\item \texttt{+ GetObjectToken(): HRESULT} metodo che serve per rendere disponibile l'oggetto \texttt{voiceToken} all'esterno;
	\begin{itemize}
		\item \texttt{ppToken: ISpObjectToken} argomento utilizzato per ritornare il l'oggetto impostato tramite il metodo \texttt{SetObjectToken()};
	\end{itemize}
	\item \texttt{Speak(): HRESULT} metodo che esegue la sintesi vocale mediante l'engine MaryTTS. Aggiunge gli eventi alla coda e gestisce l'output dell'engine tramite \texttt{pOutputSite}.\\\\
	\textbf{Argomenti}
	\begin{itemize}
		\item \texttt{dwSpeakFlags: DWORD} contiene i valori dei flags che descrivono le caratteristiche dell'input;
		\item \texttt{rguidFormatId: GUID} identificatore del formato di uscita della sintesi vocale. I possibili valori possono essere \texttt{SPDFID\_Text} o\\\texttt{SPDFID\_WAVEFORMATEX};
		\item \texttt{pWaveFormatEx: WAVEFORMATEX} puntatore alla struttura che descrive il formato d'uscita se il parametro \texttt{rguidFormatId} ha valore\\\texttt{SPDFID\_WAVEFORMATEX}. L'argomento ha valore \texttt{NULL} se il parametro \texttt{rguidFormatId} ha valore \texttt{SPDFID\_Text};
		\item \texttt{pTextFragList: SPVTEXTFRAG} lista concatenata di \texttt{SPVTEXTFRAG} su cui eseguire la sintesi vocale. Un elemento \texttt{SPVTEXTFRAG} è formato da un frammento di testo decorato da altri attributi che ne descrivono meglio le caratteristiche;
		\item \texttt{pOutputSite: ISpTTSEngineSite} è il puntatore all'interfaccia\\\texttt{ISpTTSEngineSite} che viene utilizzato per scrivere l'audio nel buffer dedicato e aggiungere gli eventi SAPI alla coda.
	\end{itemize}
	\item \texttt{+ GetOutputFormat(): HRESULT} ritorna il formato audio previsto dall'engine MaryTTS;
	\\\\
	\textbf{Argomenti}
	\begin{itemize}
		\item \texttt{pTargetFmtId: GUID} id del formato richiesto come uscita. I valori possibili possono essere di due tipi: \texttt{SPDFID\_Text} per ottenere un formato testuale e \texttt{SPDFID\_WAVEFORMATEX} per un formato audio;
		\item \texttt{pTargetWaveFormatEx: WAVEFORMATEX} se l'identificatore del formato è del tipo \texttt{SPDFID\_WAVEFORMATEX} l'argomento contiene il puntatore alla struttura del formato audio, altrimenti il suo valore è \texttt{NULL};
		\item \texttt{pOutputFormatId: GUID} contiene l'identificatore del formato di uscita che può essere \texttt{SPDFID\_Text} o \texttt{SPDFID\_WAVEFORMATEX}
		\item \texttt{ppCoMemOutputWaveFormatEx: WAVEFORMATEX} contiene la struttura di tipo \texttt{WAVEFORMATEX} del formato audio se l'argomento \texttt{pOutputFormatId} è impostato al valore \texttt{SPDFID\_WAVEFORMATEX} altrimenti è \texttt{NULL}. La struttura verrà allocata tramite \texttt{CoTaskMemAlloc}.
	\end{itemize}
	\item \texttt{- ResetActions(): void} metodo che ha il compito di ripristinare l'evoluzione delle azioni compiute allo stato iniziale comportando l'interruzione della sintesi vocale.
	\item \texttt{- HandleActions(): void} metodo che permette di gestire le azioni provenienti dall'applicazione, come la regolazione del volume e della velocità di riproduzione.
	\\\\
	\textbf{Argomenti}
	\begin{itemize}
		\item \texttt{site: ISpTTSEngineSite} riferimento all'interfaccia \texttt{ISpTTSEngineSite} che permette di recuperare le azioni svolte dall'applicazione;
		\item \texttt{utterance: json\_t} oggetto che rappresenta lo stato corrente dei parametri della voce dell'engine TTS. L'oggetto contiene le informazioni relative al volume, tonalità e velocità della voce.
	\end{itemize}
	\item \texttt{- HandleEventInterests(): void} metodo che viene utilizzato all'interno della classe \texttt{FATTS} per capire la tipologia di eventi che vengono sollevati. Questo metodo è utile per variare il comportamento della classe in base agli eventi che si presentano;\\\\
	\textbf{Argomenti}
	\begin{itemize}
		\item \texttt{site: ISpTTSEngineSite} riferimento all'interfaccia \texttt{ISpTTSEngineSite} per permettere la chiamata \texttt{ISpTTSEngineSite::GetEventInterest()} in modo da recuperare la tipologia di eventi sollevata.
	\end{itemize}
	\item \texttt{AdjustProperties(): FragmentPropertiesPtr} metodo che serve per aggiornare le proprietà associate alla voce. L'aggiornamento viene effettuato solamente se sono presenti nuovi valori, altrimenti le proprietà restano invariate.
	\\\\
	\textbf{Argomenti}
	\begin{itemize}
		\item \texttt{state: SPVSTATE} rappresenta l'insieme delle proprietà associate ad un frammento del testo;
		\item \texttt{props: FragmentPropertiesPtr} mappa che contiene le proprietà della voce;
		\item \texttt{utterance: json\_t} oggetto che viene utilizzato da MaryTTS per modificare le caratteristiche della voce durante la sintesi vocale. In particolare, in questo metodo viene utilizzato per modificare la tonalità della voce.
	\end{itemize} 		
\end{itemize}
\subsection{fatts\_client}
\texttt{fatts\_client} è il modulo che comunica direttamente con l'engine MaryTTS attraverso la libreria cURL. L'interazione tra il modulo e MaryTTS avviene mediante richieste HTTP. Invece, per gestire l'output del server in formato JSON si è scelto di utilizzare la libreria Jansson.\\ 
Il modulo si compone di pochi metodi quali l'inizializzazione, la pulizia e la richiesta al server.
Il metodo che effettua la richiesta al server sfrutta il meccanismo delle \gls{callbackg}. Infatti gli argomenti del metodo non accettano solo oggetti ma anche funzioni. Questo implica che le callback potranno essere definite all'esterno e assumere vari comportamenti.
\\\\
\textbf{Metodi}
\begin{itemize}
	\item \texttt{fatts\_init(): int} metodo che esegue l'inizializzazione della libreria cURL;
	\item \texttt{fatts\_cleanup(): void} metodo che ha il compito di eseguire la pulizia degli oggetti utilizzati dal client e dalla libreria cURL;
	\item \texttt{fatts\_client\_say\_complete(): int} metodo che esegue la richiesta completa al server. Si occupa di recuperare attraverso l'engine MaryTTS i fonemi, i visemi, le informazioni legate alle parole sintetizzate e il flusso audio da riprodurre.
	Ogni operazione viene effettuata mediante una callback passata come argomento.
	\\\\
	\textbf{Argomenti}
	\begin{itemize}
		\item \texttt{conn: fatts\_client\_s} oggetto che contiene le informazioni relative alla connessione con il server;
		\item \texttt{req : fatts\_request\_s} oggetto che raccoglie tutti i parametri che servono ad effettuare la richiesta al server. I parametri in questione contengono la configurazione della voce, il formato dell'input, l'input testuale e il tipo dell'output desiderato;
		\item \texttt{event\_cb: void} callback utilizzata per aggiungere gli eventi legati ai fonemi e ai visemi alla coda degli eventi tenuta dallo standard SAPI;
		\item \texttt{event\_cb\_data: void} oggetto utilizzato per poter permettere il passaggio dei dati utili alla callback \texttt{event\_cb};
		\item \texttt{word\_event\_cb: void} callback utilizzata per costruire gli eventi legati alle parole da sintetizzare.
		\item \texttt{word\_event\_cb\_data: void} oggetto di supporto utilizzato per rendere possibile il funzionamento della callback \texttt{event\_word\_cb}.
		\item \texttt{write\_callback: size\_t} callback utilizzata per trasferire l'audio proveniente dal server nel buffer che verrà riprodotto.
		\item \texttt{stream: void} rappresenta l'oggetto che verrà utilizzato dalla callback \texttt{write\_callback} per scrivere lo stream audio nel buffer messo a disposizione dallo standard SAPI.
	\end{itemize}
\end{itemize}
\subsection{Descrizione del funzionamento}
Adesso andremo a descrivere il flusso delle operazioni effettuate dall'implementazione con un esempio.\\
Gli input che l'utente può scegliere sono:
\begin{itemize}
	\item la voce;
	\item il testo su cui effettuare la sintesi;
	\item il volume;
	\item la velocità;
	\item la tonalità della voce;
\end{itemize}
Una volta scelti questi parametri è possibile iniziare la sintesi vocale. Ricordiamo che la voce e tutti i suoi parametri sono rappresentati da un token presente nel registro di Windows.
Quindi nel momento della selezione della voce verranno invocati i metodi \texttt{FATTS:SetObjectToken()} e \texttt{FATTS:GetObjectToken()} per rendere disponibile sia all'applicazione sia all'engine MaryTTS le informazioni relative alla voce.\\
Dopo questa operazione viene invocato il metodo \texttt{FATTS::Speak()} che esegue la sintesi vocale.\\
All'interno del metodo \texttt{FATTS::Speak()} in un primo momento viene analizzato il testo proveniente dall'applicazione, vengono estratte le proprietà come la tonalità della voce e poi viene inviato al server.
Prima di effettuare la richiesta, viene invocato il metodo \texttt{FATTS::HandleActions()} che permette di recuperare i parametri mancanti come il volume e la velocità e nel medesimo frangente vengono costruiti gli oggetti utili alle callback che avranno il compito di costruire ed aggiungere gli eventi alla coda.\\
A questo punto il metodo \texttt{FATTS::fatts\_client\_say\_complete()} è pronto per essere chiamato ed eseguire la richiesta al server.
Il metodo sfruttando la libreria cURL e Jansson elabora l'output del server in formato JSON, esegue la costruzione degli eventi e invia lo stream audio alla riproduzione.

\newpage
\section{Implementazione Interfaccia SAPI~5 per Speect}
Questa implementazione sfrutta l'engine TTS Speect fornito e sviluppato da Mivoq. Esso è distribuito sotto forma di libreria in modo da permettere il suo utilizzo da parte di altri software.
L'implementazione è analoga a quella usata per MaryTTS. Vengono implementate le interfacce SAPI~5 \texttt{ISpObjectWithToken} e \texttt{ISpTTSEngine}, inoltre vengono ereditate le classi \texttt{CComObjectRootEx} e\\\texttt{CComCoClass}.\\
Le classi e i moduli che verranno utilizzati sono: \texttt{SpeectTTS} e \texttt{audio\_event\_functions}.
\texttt{SpeectTTS} ha il compito di implementare i metodi messi a disposizione dall'interfaccia SAPI~5, invece il modulo \texttt{audio\_event\_functions} si preoccupa di interagire con l'engine TTS Speect.
\subsection{SpeectTTS}
La classe \texttt{SpeectTTS} è stata progettata per implementare i metodi dell'interfaccia SAPI~5 e per fornire la struttura minima per costruire la libreria dinamica.
Attraverso i suoi metodi la classe inizializza l'engine Speect, ne configura la voce attraverso il metodo \texttt{SpeectTTS::SetObjectToken()} e carica il plugin per gestire lo stream audio.\\
\texttt{SpeectTTS} inoltre si occupa di gestire le azioni provenienti dall'applicazione e di impostare le proprietà relative alla voce attraverso i metodi\\\texttt{SpeectTTS::HandleActions()} e \texttt{SpeectTTS::AdjustProperties()}.
Per effettuare la sintesi vocale viene invocato il metodo \texttt{SpeectTTS::Speak()} che attraverso il modulo\\\texttt{audio\_event\_functions} comunica l'engine Speect.\\\\
\textbf{Implementa:}
\begin{itemize}
	\item \texttt{ISpObjectWithToken};
	\item \texttt{ISpTTSEngine}.
\end{itemize}
\textbf{Eredita da:}
\begin{itemize}
	\item \texttt{CComObjectRootEx};
	\item \texttt{CComClass}.
\end{itemize}
\textbf{Attributi}
\begin{itemize}
	\item \texttt{- voiceToken: CComPtr<ISpObjectToken>} rappresenta l'oggetto che mantiene le informazioni relative alla voce;
	\item \texttt{- voiceProperties: FragmentPropertiesPtr} mappa che contiene le proprietà relative alla voce;
	\item \texttt{- actionAborted: boolean} flag che viene utilizzato per interrompere la sintesi vocale;
	\item \texttt{- actionSkipSentences: int} indica il numero di frasi che la sintesi vocale deve saltare durante l'elaborazione;
	\item \texttt{- error: s\_erc} oggetto che tiene traccia degli errori di Speect;
	\item \texttt{- riffAudio: SPlugin} rappresenta il plugin che ha il compito di gestire l'output audio di Speect;
	\item \texttt{- voice: SVoice} è l'oggetto che rappresenta la configurazione della voce che utilizzerà Speect per sintetizzare il testo;   
\end{itemize}
\textbf{Metodi}
\begin{itemize}
	\item \texttt{+ SpeectTTS()} è costruttore della classe che serve ad inizializzare le proprietà legate alla voce e gli oggetti che verranno utilizzati da Speect;
	\item \texttt{+ \char`\~SpeectTTS()} è il distruttore che ha il compito di eseguire la pulizia della memoria dinamica allocata dalla classe e da Speect; 
	\item \texttt{+ FinalConstruct(): HRESULT} override del metodo ereditato dalla classe \texttt{CComObjectRootEx} che serve per eseguire le inizializzazioni necessarie dell'oggetto;
	\item \texttt{+ FinalRelease(): void} override del metodo ereditato dalla classe\\\texttt{CComObjectRootEx} che serve per eseguire la pulizia dell'oggetto prima di distruggerlo;
	\item \texttt{+ SetObjectToken(): HRESULT} metodo che serve per recuperare le informazioni relative alla voce scritte nel registro di Windows. Questo metodo andrà ad impostare l'attributo \texttt{voiceToken} e inizializzerà Speect.\\\\
	\textbf{Argomenti}
	\begin{itemize}
		\item \texttt{pToken: ISpObjectToken} è il riferimento al token della voce contenuto nel registro di Windows. 
	\end{itemize}
	\item \texttt{+ GetObjectToken(): HRESULT} metodo che serve per rendere disponibile l'oggetto \texttt{voiceToken} all'esterno;
	\begin{itemize}
		\item \texttt{ppToken: ISpObjectToken} argomento utilizzato per ritornare il l'oggetto impostato tramite il metodo \texttt{SetObjectToken()};
	\end{itemize}
	\item \texttt{Speak(): HRESULT} metodo che serve per effettuare la sintesi vocale attraverso l'engine Speect. Aggiunge gli eventi alla coda e gestisce l'output dell'engine mediante \texttt{pOutputSite}.\\\\
	\textbf{Argomenti}
	\begin{itemize}
		\item \texttt{dwSpeakFlags: DWORD} contiene i valori dei flags che descrivono le caratteristiche dell'input;
		\item \texttt{rguidFormatId: GUID} identificatore del formato di uscita della sintesi vocale. I possibili valori possono essere \texttt{SPDFID\_Text} o\\\texttt{SPDFID\_WAVEFORMATEX};
		\item \texttt{pWaveFormatEx: WAVEFORMATEX} puntatore alla struttura che descrive il formato d'uscita se il parametro \texttt{rguidFormatId} ha valore\\\texttt{SPDFID\_WAVEFORMATEX}. L'argomento ha valore \texttt{NULL} se il parametro \texttt{rguidFormatId} ha valore \texttt{SPDFID\_Text};
		\item \texttt{pTextFragList: SPVTEXTFRAG} lista concatenata di \texttt{SPVTEXTFRAG} su cui eseguire la sintesi vocale. Un elemento \texttt{SPVTEXTFRAG} è formato da un frammento di testo decorato da altri attributi che ne descrivono meglio le caratteristiche;
		\item \texttt{pOutputSite: ISpTTSEngineSite} è il puntatore all'interfaccia\\\texttt{ISpTTSEngineSite} che viene utilizzato per scrivere l'audio e aggiungere gli eventi SAPI alla coda.
	\end{itemize}
	\item \texttt{+ GetOutputFormat(): HRESULT} ritorna il formato audio previsto dall'engine Speect;
	\\\\
	\textbf{Argomenti}
	\begin{itemize}
		\item \texttt{pTargetFmtId: GUID} id del formato richiesto come uscita. I valori possibili possono essere di due tipi: \texttt{SPDFID\_Text} per ottenere un formato testuale e \texttt{SPDFID\_WAVEFORMATEX} per un formato audio;
		\item \texttt{pTargetWaveFormatEx: WAVEFORMATEX} se l'identificatore del formato è del tipo \texttt{SPDFID\_WAVEFORMATEX} l'argomento contiene il puntatore alla struttura del formato audio, altrimenti il suo valore è \texttt{NULL};
		\item \texttt{pOutputFormatId: GUID} contiene l'identificatore del formato di uscita che può essere \texttt{SPDFID\_Text} o \texttt{SPDFID\_WAVEFORMATEX}
		\item \texttt{ppCoMemOutputWaveFormatEx: WAVEFORMATEX} contiene la struttura di tipo \texttt{WAVEFORMATEX} del formato audio se l'argomento \texttt{pOutputFormatId} è impostato al valore \texttt{SPDFID\_WAVEFORMATEX} altrimenti è \texttt{NULL}. La struttura verrà allocata tramite \texttt{CoTaskMemAlloc}.
	\end{itemize}
	\item \texttt{- ResetActions(): void} metodo che ha il compito di ripristinare l'evoluzione delle azioni compiute allo stato iniziale comportando l'interruzione della sintesi vocale.
	\item \texttt{- HandleActions(): void} metodo che permette di gestire le azioni provenienti dall'applicazione, come la regolazione del volume e della velocità di riproduzione.
	\\\\
	\textbf{Argomenti}
	\begin{itemize}
		\item \texttt{site: ISpTTSEngineSite} riferimento all'interfaccia \texttt{ISpTTSEngineSite} per permettere il recupero delle azioni svolte dall'applicazione;
		\item \texttt{in\_utterance: void} oggetto che rappresenta l'\gls{utteranceg} corrente e  viene modificato in base alle azioni che vengono compiute. Ad esempio ad una variazione del volume, l'oggetto in questione viene modificato per permettere la medesima regolazione all'interno dell'engine Speect;
	\end{itemize}
	\item \texttt{- HandleEventInterests(): void} metodo che viene utilizzato all'interno della classe \texttt{SpeectTTS} per capire la tipologia di eventi che vengono sollevati. Questo metodo è utile per variare il comportamento della classe in base agli eventi che si presentano;\\\\
	\textbf{Argomenti}
	\begin{itemize}
		\item \texttt{site: ISpTTSEngineSite} riferimento all'interfaccia \texttt{ISpTTSEngineSite} per permettere la chiamata \texttt{ISpTTSEngineSite::GetEventInterest()} in modo da recuperare la tipologia di eventi sollevata.
	\end{itemize}
	\item \texttt{AdjustProperties(): FragmentPropertiesPtr} metodo che serve per aggiornare le proprietà associate alla voce. L'aggiornamento viene effettuato solamente se sono presenti nuovi valori, altrimenti le proprietà restano invariate.
	\\\\
	\textbf{Argomenti}
	\begin{itemize}
		\item \texttt{state: SPVSTATE} rappresenta l'insieme delle proprietà associate ad un frammento del testo;
		\item \texttt{props: FragmentPropertiesPtr} mappa che contiene le proprietà della voce;
		\item \texttt{utterance: void} oggetto che viene utilizzato da Speect per modificare le caratteristiche della voce durante la sintesi vocale. In particolare, in questo metodo viene utilizzato per modificare la tonalità della voce.
	\end{itemize} 		
\end{itemize}
\subsection{audio\_event\_functions}
\texttt{audio\_event\_functions} è il modulo che si occupa di interagire con l'engine Speect mediante chiamate a funzioni. 
Il metodo principale che compone il modulo è \texttt{speect\_request\_complete\_text()}. I compiti del metodo sono: scorrere il grafo Heterogeneous Relation Graph (HRG) presente all'interno di Speect, recuperare le informazioni dell'input relative alle parole e ai fonemi, aggiungere gli eventi SAPI alla coda e scrivere l'audio generato nel buffer gestito da SAPI.\\
Per adempire ai task relativi allo standard SAPI il modulo \texttt{audio\_event\_functions} fa utilizzo di callback definite nella classe \texttt{SpeectTTS}.
\\\\
\textbf{Metodi}
\begin{itemize}
	\item \texttt{speect\_request\_complete\_text(): int} metodo che attraverso l'utilizzo di callback passate come argomento si occupa di gestire gli eventi SAPI relativi alle parole, ai fonemi e ai visemi e di scrivere l'output audio fornito da Speect nel buffer gestito dal sistema operativo.
	\\\\
	\textbf{Argomenti}
	\begin{itemize}
		\item \texttt{input: void} oggetto che contiene i parametri utili a Speect per eseguire la sintesi vocale;
		\item \texttt{event\_cb : void} callback utilizzata per aggiungere gli eventi legati ai fonemi e ai visemi alla coda degli eventi tenuta dallo standard SAPI;
		\item \texttt{event\_cb\_data: void} oggetto contenente i dati utili alla callback \texttt{event\_cb};
		\item \texttt{word\_event\_cb: void} callback utilizzata per costruire gli eventi legati alle parole da sintetizzare.
		\item \texttt{word\_event\_cb\_data: void} oggetto di supporto utilizzato per rendere possibile il funzionamento della callback \texttt{event\_word\_cb}.
		\item \texttt{write\_callback: size\_t} callback utilizzata per trasferire l'audio proveniente dal server nel buffer che verrà riprodotto.
		\item \texttt{stream: void} rappresenta l'oggetto che verrà utilizzato dalla callback \texttt{write\_callback} per scrivere lo stream audio nel buffer messo a disposizione dallo standard SAPI.
	\end{itemize}
\end{itemize}
\subsection{Descrizione del funzionamento}
Adesso andremo a descrivere il flusso delle operazioni effettuate dall'implementazione con un esempio.
Gli input che l'utente può scegliere tipicamente sono:
\begin{itemize}
	\item la voce;
	\item il testo su cui effettuare la sintesi;
	\item il volume;
	\item la velocità;
	\item la tonalità della voce;
\end{itemize}
Una volta scelti questi parametri è possibile iniziare la sintesi vocale. Ricordiamo che la voce e tutti i suoi parametri sono rappresentati da un token presente nel registro di Windows.
Quindi, nel momento della selezione della voce verranno invocati i metodi \texttt{SpeectTTS:SetObjectToken()} e \texttt{SpeectTTS:GetObjectToken()} per rendere disponibile sia all'applicazione sia all'engine Speect le informazioni relative alla voce.\\
Dopo questa operazione viene invocato il metodo \texttt{SpeectTTS::Speak()} che esegue la sintesi vocale.\\
In un primo momento, all'interno del metodo \texttt{SpeectTTS::Speak()} viene analizzato il testo proveniente dall'applicazione, vengono estratte le proprietà come la tonalità della voce e poi l'input  viene inviato a Speect per essere processato.\\
Prima di eseguire la sintesi, viene invocato il metodo \texttt{HandleActions()} che permette di recuperare i parametri mancanti come il volume e la velocità e nel medesimo frangente vengono costruiti gli oggetti utili alle callback che avranno il compito di costruire ed aggiungere gli eventi alla coda.
A questo punto il metodo \texttt{speect\_request\_complete\_text()} è pronto per essere chiamato ed eseguire la sintesi vocale attraverso Speect.
Durante questa chiamata, Speect scorre le relazioni all'interno del suo Heterogeneous Relation Graph, recuperando le informazioni relative alle parole e ai fonemi. Grazie ad esse riesce attraverso le callback a costruire gli eventi SAPI associati.\\
Infine Speect mediante il plugin dedicato al rendering audio, genera l'output e attraverso la callback lo carica nel buffer gestito dal sistema operativo.

\newpage
\section{MaryTTS}
MaryTTS è un engine TTS multilingua open-source sviluppato da DFKI. Dopo aver valutato le sue caratteristiche e funzionalità Mivoq ha scelto di contribuire al suo sviluppo e di integrarlo nella piattaforma FA-TTS.
L'azienda attraverso il suo lavoro ha esteso le funzionalità originali di MaryTTS fornendo nuove API e aggiungendo nuovi formati di input ed output.\\
La piattaforma FA-TTS, grazie al contributo di Mivoq, è grado di eseguire la sintesi vocale con qualsiasi voce. Questo è reso possibile, perché l'azienda mette a disposizione un servizio che permette di digitalizzare ogni tipo voce.
\subsection{Interfaccia Web Client}
La piattaforma FA-TTS è scritta in Java e può essere installata nel sistema come applicazione server.
Dispone di un'interfaccia web molto semplice composta da una casella di testo dove inserire l'input da sintetizzare e delle combo box che servono a selezionare la voce, il tipo di input e il tipo di output desiderati.
Nel caso in cui l'output fosse di tipo audio è possibile selezionare il formato e impostare alcuni effetti audio, tra cui il volume, la tonalità e la velocità della voce.
\subsection{FA-TTS REST API}
La piattaforma FA-TTS mette a disposizione delle API che possono essere interrogate tramite richieste HTTP.
Le API più rilevanti sono:
\begin{itemize}
	\item \textbf{/say} viene utilizzata per eseguire la sintesi vocale in base ai parametri richiesti;
	\item \textbf{/info/version} serve a recuperare la versione corrente del server;
	\item \textbf{/info/voices/all} serve a recuperare tutte le voci disponibili nel server.
\end{itemize}
Adesso andremo ad analizzare degli esempi di richieste utili all'implementazione SAPI per MaryTTS.
\subsubsection{Richiesta stream audio}
Per eseguire la richiesta dello stream audio si utilizza l'API \textbf{/say} nella seguente forma:
\lstset{language=html,
		basicstyle=\ttfamily,
		columns=fullflexible,
		showstringspaces=false,
		breaklines=true }
\begin{lstlisting}
http://localhost:59125/say?input[type]=TEXT&input[content]=Benvenuti+nel+mondo+della+sintesi+vocale.&input[locale]=it&output[type]=AUDIO&output[format]=WAVE_FILE&voice[gender]=male&voice[name]=istc-speaker_internazionale-hsmm&voice[age]=35&voice[variant]=1&voice[selection_algorithm]=ssml&utterance[effects]=[{"Volume":1.5},{"F0Add":50.0},{"Rate":3.0}]
\end{lstlisting}
Ad esempio questa richiesta permette di sintetizzare il testo "Benvenuti nel mondo della sintesi vocale" con la voce "istc-speaker\_internazionale-hsmm" di genere maschile con 35 anni.
Sono stati aggiunti anche degli effetti come l'amplificazione del volume, il cambio di tonalità e la diminuzione della velocità.
\subsubsection{Richiesta parole e fonemi}
La seguente richiesta viene impiegata per recuperare le parole e i fonemi sintetizzati da MaryTTS.
\begin{lstlisting}
	http://localhost:59125/say?input[type]=TEXT&input[content]=Benvenuti+nel+mondo+della+sintesi+vocale.&input[locale]=it&output[type]=LIPSYNC&voice[gender]=male&voice[name]=istc-speaker_internazionale-hsmm&voice[age]=35&voice[variant]=1&voice[selection_algorithm]=ssml&utterance[effects]=[{"Volume":1.5},{"F0Add":50.0},{"Rate":3.0}]
\end{lstlisting}
La richiesta è molto simile a quella per recuperare lo stream audio, l'unica differenza è il parametro \texttt{output[type]} impostato a \texttt{LIPSYNC}.\\
La risposta del server è in formato JSON ed è costruita nel seguente modo:
\lstset{language=html,
	basicstyle=\ttfamily,
	columns=fullflexible,
	showstringspaces=false,
	breaklines=true }
\begin{lstlisting}
{
"tokens":[{ 
			"char_end": integer, 
			"start": float,
			"end": float,
			"char_start": integer
}],
"segments":[{ 
			"stressed": boolean, 
			"start": float,
			"end": float,
			"label":  string
}]
}

\end{lstlisting}
L'output è formato da un oggetto composto da due array: \texttt{tokens} e \texttt{segments}.\\
L'array \texttt{tokens} serve a recuperare le informazioni relative alle parole sintetizzate.
Per ogni parola o token è possibile recuperare, grazie a questo array, l'offset espresso in caratteri rispetto all'inizio e alla fine del testo e l'offset espresso in secondi di inizio e fine rispetto allo stream audio.\\
L'array \texttt{segments} serve per recuperare le informazioni relative ai fonemi sintetizzati.
Per ogni fonema è possibile recuperare l'offset di inizio e fine rispetto allo stream audio, l'etichetta associata al fonema e la proprietà di stress.

\newpage
\section{Speect}
Speect è un engine TTS multilingua sviluppato dal gruppo Human Language Technologies del Meraka Institute in Sud Africa. 
Mivoq ha scelto di utilizzare questo engine e di portarne avanti lo sviluppo per due motivi essenziali:
\begin{itemize}
	\item Speect è un engine TTS completo e composto interamente da plugin. Questo favorisce un'alta modularità e aumenta la capacità di testing;
	\item Speect è scritto interamente in C seguendo lo standard ISO/IEC 9899:1990. Questo permette di aver una alta portabilità e compatibilità anche con sistemi non recenti. 
\end{itemize}
La peculiarità di Speect è che il suo funzionamento si basa su una configurazione chiamata voce. Questo permette di scegliere, ad esempio, quali plugin devo essere abilitati e le caratteristiche che deve avere la sintesi.
Speect può essere utilizzato dalle applicazioni esterne e dai sistemi operativi sotto forma di libreria. Nel caso del sistema operativo Microsoft Windows, Speect viene fornito come libreria dinamica (DLL).
\subsection{Heterogeneous Relation Graph}
Speect per gestire l'analisi testuale sfrutta un grafo chiamato Heterogeneous Relation Graph.
Il grafo ha il compito di tenere traccia delle relazioni che ci sono all'interno del testo.
Le relazioni sono:
\begin{itemize}
	\item \textbf{Phrase} gestisce le relazioni che esistono fra le varie frasi. Tra una frase e l'altra sono presenti due archi, uno punta all'elemento precedente e uno al successivo. Inoltre ogni elemento è il padre del corrispondente elemento nella relazione Word, che da accesso alle singole parole.   
	\item \textbf{Token} gestisce le relazioni che esistono tra i vari tokens. I tokens rappresentano delle unità di testo indivisibili a livello logico.
	Ogni token è collegato tramite due archi, il successivo e il precedente.
	Ogni elemento della relazione è il padre del corrispondente elemento nella relazione Word.
	\item \textbf{Word} gestisce le relazioni che esistono tra le varie parole. Ogni parola è connessa tramite gli archi precedente e successivo.
	\item \textbf{Syllable} gestisce le relazioni tra le varie sillabe. Ogni sillaba è connessa tramite gli archi precedente e successivo.
	\item \textbf{Segment} gestisce le relazioni tra i vari fonemi. Ogni fonema è connesso tramite gli archi precedente e successivo.
	\item \textbf{SylStructure} è l'unione delle relazioni Word, Syllable e Segment, tramite l'arco genitore e figlia.
\end{itemize}
Tutte le relazioni possono essere attraversate orizzontalmente tramite le operazioni \textbf{Next} e \textbf{Previous}.
Solamente le relazioni Phrase, Token e SylStructure possono essere attraversate tramite le operazioni \textbf{Parent} e \textbf{Daughter}.
%immagine grafo HRG                                                                                                 
Il grafo HRG è stato fondamentale per la costruzione degli eventi SAPI legati ai fonemi, ai visemi e alle parole.
\subsection{Abilitare la serializzazione di un oggetto SAudio in un buffer qualsiasi}
Per soddisfare il requisito \textbf{ob02} sono state aggiunte nuove funzionalità all'interno di Speect.
La necessità di quest'operazione derivava dal fatto che l'engine TTS non era in grado di serializzare un oggetto di tipo \texttt{SAudio} in un buffer.
L'unica funzione a disposizione era \texttt{SObjectSave()} che permetteva di serializzare un oggetto \texttt{SAudio} come un file dato un percorso.\\
Per ovviare al problema sono stati introdotti una nuova classe chiamata\\\texttt{SGenericsource} e dei nuovi metodi.
\subsubsection{Classe SGenericsource}
La classe \texttt{SGenericsource} estende la classe base \texttt{SDatasource} e ha il compito associare ad un oggetto qualsiasi delle operazioni di input-output in modo da permettere la gestione completa del comportamento dell'oggetto.\\\\
\textbf{Eredita da:}
\begin{itemize}
	\item \texttt{SDatasource};
\end{itemize}
\textbf{Attributi}
\begin{itemize}
	\item \texttt{- ptr: void} riferimento all'oggetto generico che deve essere trattato tramite le funzioni di input-output definite all'esterno; 
	\item \texttt{- io\_functs: SIOFunctions} oggetto che contiene le funzioni di input-output definite all'esterno. Le funzioni sono \texttt{read}, \texttt{write}, \texttt{seek} e \texttt{close}.  
\end{itemize}
\textbf{Metodi}
\begin{itemize}
	\item \texttt{+ \_s\_generic\_source\_class\_add(): void} metodo che aggiunge la classe \texttt{SGenericsource} al sistema degli oggetti di Speect.\\
	\textbf{Argomenti}
	\begin{itemize}
		\item \texttt{error: s\_erc} oggetto che tiene traccia degli errori generati da Speect.
	\end{itemize}
	\item \texttt{+ SGenericsourceOpen(): SDatasource} metodo che converte un oggetto qualsiasi nel tipo \texttt{SDatasource}. Assieme all'oggetto devono essere fornite anche le funzioni di input-output.\\
	\textbf{Argomenti}
	\begin{itemize}
		\item \texttt{ptr: void} oggetto che deve essere interpretato come \texttt{SDatasource};
		\item \texttt{mode: char} stringa che indica in quale modalità deve essere trattato l'oggetto;
		\item \texttt{io\_functs: SIOFunctions} oggetto che raccoglie le funzioni di input e output che servono ad eseguire le operazioni sull'oggetto \texttt{ptr};
		\item \texttt{error: s\_erc} oggetto che tiene traccia degli errori generati da Speect.
	\end{itemize} 
\end{itemize}

\subsubsection{Modulo serialize}
Il modulo \texttt{serialize} è una raccolta di funzioni che si occupa della \gls{serializzazioneg} degli oggetti trattati da Speect attraverso l'\gls{objectsystemg}.
Per soddisfare il requisito \textbf{ob02} è stato aggiunto un nuovo metodo chiamato \texttt{SObjectSaveToDatasource()}.\\\\
\textbf{Metodi:}
\begin{itemize}
	\item \texttt{+ SObjectSaveToDatasource(): void} serve a serializzare un oggetto che appartiene all'Object System di Speect in un oggetto di tipo \texttt{SDatasource}.\\\\
	\textbf{Argomenti}
	\begin{itemize}
		\item \texttt{object: SObject} oggetto che deve essere serializzato;
		\item \texttt{ds: SDatasource} destinazione della serializzazione dell'oggetto;
		\item \texttt{format: char} formato di destinazione dell'oggetto che deve essere serializzato;
		\item \texttt{error: s\_erc} oggetto che traccia gli errori generati da Speect.
	\end{itemize}
\end{itemize}

\subsubsection{Classe SSerializedFileClass}
\texttt{SSerializedFileClass} è una classe astratta che estende \texttt{SObjectClass} e viene utilizzata per trattare i file e le sue operazioni di input e output.
In origine la classe permetteva tramite una sua estensione concreta di serializzare un oggetto solo in un percorso. Per ovviare a questa limitazione si è scelto di introdurre un nuovo metodo chiamato \texttt{SSerializedFileSaveToDatasource()}.\\\\
\textbf{Eredita da:}
\begin{itemize}
	\item \texttt{SObjectClass}.
\end{itemize}
\textbf{Metodi:}
\begin{itemize}
	\item \texttt{+ SSerializedFileSaveToDatasource(): void} serve a serializzare un oggetto che appartiene all'Object System di Speect in un oggetto di tipo \texttt{SDatasource}.\\
	\textbf{Argomenti}
	\begin{itemize}
		\item \texttt{self: SSerializedFile} oggetto che viene utilizzato come riferimento interno per invocare i metodi della classe; 
		\item \texttt{object: SObject} oggetto che deve essere serializzato;
		\item \texttt{ds: SDatasource} destinazione della serializzazione dell'oggetto;
		\item \texttt{format: char} formato di destinazione dell'oggetto che deve essere serializzato;
		\item \texttt{error: s\_erc} oggetto che traccia gli errori generati da Speect.
	\end{itemize}
\end{itemize}
 
\subsubsection{Classe SRIFFAudioFileClass}
La classe \texttt{SRIFFAudioFileClass} è un'estensione concreta della classe astratta \texttt{SSerializedFileClass} e ha il compito di serializzare un oggetto di tipo \texttt{SAudio} nel formato \gls{riffg}.
Per permettere la serializzazione verso un oggetto di tipo \texttt{SDatasource} è stato aggiunto un nuovo metodo chiamato \texttt{SaveToDatasource()}.\\\\
\textbf{Eredita da:}
\begin{itemize}
	\item \texttt{SSerializedFileClass}.
\end{itemize}
\textbf{Metodi}
\begin{itemize}
	\item \texttt{+ SaveToDatasource(): void} serve a serializzare un oggetto di tipo \texttt{SAudio} in un oggetto di tipo \texttt{SDatasource}.\\\\
	\textbf{Argomenti}
	\begin{itemize}
		\item \texttt{object: SObject} oggetto che deve essere serializzato;
		\item \texttt{ds: SDatasource} destinazione della serializzazione dell'oggetto;
		\item \texttt{error: s\_erc} oggetto che traccia gli errori generati da Speect.
	\end{itemize}
\end{itemize}

\subsubsection{Modulo write}
Il modulo \texttt{write} fornisce delle funzioni di supporto alla classe \texttt{SRIFFAudioFileClass}. Per implementare il metodo \texttt{SRIFFAudioFileClass::SaveToDatasource()} è stato necessario aggiungere un nuovo metodo al modulo \texttt{write} chiamato\\\texttt{s\_write\_audio\_riff\_16\_to\_datasource()}.\\\\
\textbf{Metodi}
\begin{itemize}
	\item \texttt{+ s\_write\_audio\_riff\_16\_to\_datasource(): void} serve a serializzare un oggetto di tipo \texttt{SAudio} in un oggetto di tipo \texttt{SDatasource}.\\\\
	\textbf{Argomenti}
	\begin{itemize}
		\item \texttt{object: SObject} oggetto che deve essere serializzato;
		\item \texttt{ds: SDatasource} destinazione della serializzazione dell'oggetto;
		\item \texttt{error: s\_erc} oggetto che traccia gli errori generati da Speect.
	\end{itemize}
\end{itemize}

\subsection{Aggiunta feature cambio volume, velocità e tonalità}
Speect, come richiesto dal requisito \textbf{de01}, deve supportare il cambio di velocità e di tonalità della voce. 
Per soddisfare questo requisito sono stati aggiunti dei nuovi metodi all'interno del plugin che gestisce l'output audio.
Il plugin in questione utilizza le API HTS\_engine 1.05 che sono in grado di generare e gestire uno stream audio.\\
Di seguito andremo ad analizzare i metodi che sono stati aggiunti.\\\\
\textbf{Metodi:}
\begin{itemize}
	\item \texttt{check\_and\_change\_rate\_volume()} serve a modificare il volume o la velocità della voce se è stato rilevato un cambiamento;\\\\
	\textbf{Argomenti}
	\begin{itemize}
		\item \texttt{HTSsynth: SHTSEngineSynthUttProc105} oggetto che rappresenta \\l'\gls{utteranceprocg} associato al plugin che utilizza le API HTS\_engine 1.05;
		\item \texttt{utt: SUtterance} oggetto che rappresenta l'utterance associata alla sintesi corrente;
		\item \texttt{error: s\_erc} oggetto che tiene traccia degli errori di Speect.
	\end{itemize}
	\item \texttt{check\_and\_change\_tone()} serve a modificare la tonalità della voce se è stato rilevato un cambiamento;\\\\
	\textbf{Argomenti}
	\begin{itemize}
		\item \texttt{HTSsynth: SHTSEngineSynthUttProc105} oggetto che rappresenta l'Utterance Processor associato al plugin che utilizza le API HTS\_engine 1.05;
		\item \texttt{utt: SUtterance} oggetto che rappresenta l'utterance associata alla sintesi corrente;
		\item \texttt{error: s\_erc} oggetto che tiene traccia degli errori di Speect.
	\end{itemize}
\end{itemize}

