\chapter{Verifica e Validazione}
La verifica e la validazione permettono di tenere sotto controllo la qualità del software.
La verifica viene effettuata per individuare problemi o per verificare che le funzionalità siano state implementate.\\
La validazione, invece, viene svolta dal committente e verifica tramite delle prove che i requisiti siano stati soddisfatti.\\
In questo capitolo verranno riportati i risultati ottenuti, i problemi riscontrati e le impressioni fornite dal committente. 
\thispagestyle{empty}

\newpage
\section{Verifica}
Vista la natura sperimentale del progetto, non c'è stata la necessità di introdurre dei test automatici.\\
La strada che si è percorsa è stata quella di utilizzare due applicazioni per effettuare dei test.\\
Le applicazioni utilizzate sono state:
\begin{itemize}
	\item TTSApplication;
	\item Balabolka;
\end{itemize}
I test sono stati condotti principalmente seguendo due fasi:
\begin{itemize}
	\item \textbf{prima fase} consisteva nel verificare che dato un input testuale con dei parametri di velocità e di tonalità della voce di default venisse effettuata la sintesi vocale.
	\item \textbf{seconda fase} consisteva nel verificare tramite log la corretta sequenza degli eventi SAPI lanciati dall'applicazione.
\end{itemize}
In futuro questa procedura potrà essere sostituita o automatizzata e migliorata per consentire uno sviluppo di qualità maggiore.
Per quanto riguarda il requisito \textbf{de02}, che riguarda i test di unità relativi a Speect, non è stato soddisfatto, perchè ci sono state delle complicazioni riguardo la compilazione del plugin Hunpos all'interno dell'ambiente Windows.\\
Vista l'importanza del plugin all'interno di Speect, si è preferito dedicare del tempo alla sua corretta compilazione, invece che introdurre nuovi test di unità.\\
Il requisito \textbf{de02} non è stato soddisfatto perchè in corso d'opera è stato deciso di dedicare maggior tempo al requisito \textbf{op01}, che ha permesso di sviluppare quasi completamente la gestione degli eventi SAPI.
\section{Validazione}
Negli ultimi giorni di stage è stato condotto, assieme al tutor aziendale, il collaudo.
I risultati ottenuti sono stati giudicati positivi, perchè quasi tutti i requisiti sono stati soddisfatti.\\
Quelli che risultavano non soddisfatti o parzialmente soddisfatti sono:
\begin{itemize}
	\item \textbf{de02}
	\item \textbf{op01}
	\item \textbf{op02}
\end{itemize}
Il tutor aziendale ha eseguito il collaudo tramite le applicazioni utilizzate per i test, ovvero TTSApp e Balabolka.
Durante il collaudo non sono stati evidenziati gravi problemi riguardanti il progetto, ma sono emerse delle carenze da parte degli engine TTS utilizzati rispetto allo standard SAPI~5.
Infatti quando le applicazioni utilizzate per i test facevano uso degli engine TTS si è notato che durante la sintesi vocale le parole non venivano evidenziate correttamente.
Questo problema è emerso sia con l'utilizzo di MaryTTS che con Speect.\\
Nel primo caso le parole non venivano evidenziate correttamente perchè MaryTTS non generava l'array \texttt{tokens} quando incontrava la sequenza ". lettera\_maiuscola" nell'input testuale.\\
Nel secondo caso, Speect considerava l'offset e la lunghezza delle parole in caratteri e non in byte. Questo comportava che l'offset e la lunghezza dei caratteri utilizzati negli eventi di tipo \texttt{SPEI\_WORD\_BOUNDARY} non fossero corretti.\\
Tralasciando questi due aspetti legati agli engine, che verranno risolti in futuro, il tutor aziendale è rimasto molto soddisfatto del lavoro svolto.

\section{Sistema utilizzato per lo sviluppo}
Lo sviluppo del progetto di stage e i test sono stati condotti nel seguente sistema:
\begin{itemize}
	\item \textbf{Sistema operativo} Microsoft Windows 10
	\item \textbf{Processore} Intel i7-6700HQ
\end{itemize}


